%******************************************************************************%
%                                                                              %
%                  instructions.tex for 42's Piscine C++                       %
%                  Created on : Mon Sep  8 15:57:19 2014                       %
%                  Made by : David "Thor" GIRON <thor@42.fr>                   %
%                                                                              %
%******************************************************************************%


\chapter{General rules}


    \begin{itemize}

		\item Qualsiasi funzione implementata in un header(fatta eccezione per i templates),
		e qualsiasi header non protetto, porterà ad uno 0 per l'esercizio.

		\item Ogni output dovrà essere indirizzato allo standard output, e terminato da una
		nuova riga, a meno che non sia specificato diversamente.

		\item I nomi dei file imposti dovranno essere seguiti alla lettera,
		ls stessa cosa vale per i nomi delle classi, delle funzioni e dei metodi.

		\item Ricorda: Stai programmando in \texttt{C++} ora, non più in
          \texttt{C}. Quindi:
		  
		  \begin{itemize}
		  
		  \item	Le seguenti funzioni sono PROIBITE ed il loro utilizzo
		  porterà inevitabilmente ad uno \texttt{0}, niente domande: \texttt{*alloc},
		  \texttt{*printf} e \texttt{free}.
		  

		  \item Sei autorizzato ad utilizzare tutto ciò che si trova nella libreria
		  standard. COMUNQUE, sarebbe saggio se provassi ad utilizzare 
		  la versione proto-C++ delle funzioni che sei abituato ad usare in C,
		  invece di restare fermo a quello che già conosci, questa è una nuova lingua.
		  E NO, non sei autorizzato ad utilizzare STL fino a che non dovrai utilizzarlo(
		  module 08). Ciò significa niente vettori/liste/mappe/etc... o qualsivoglia cosa
		  richieda un include <algorithm> fino a quel momento.
		  

		  \end{itemize}
		
		\item L'utilizzo di qualsiasi funzione o meccanica esplicitamente
		proibita porterà ad uno \texttt{0}.

	\item Tieni anche a mente che se non è specificato diversamente, 
		le parole chiave del \texttt{C++}  \texttt{"using namespace"} e \texttt{"friend"}
		sono proibite. Il loro utilizzo verrà punito con un \texttt{-42}, senza appello.

	\item I file associati con una classe saranno sempre
          \texttt{ClassName.hpp} e \texttt{ClassName.cpp}, a meno
          di specifica diversa.

	\item Le cartelle di consegna sono \texttt{ex00/}, \texttt{ex01/},
          \dots, \texttt{exn/}.
       

	\item Devi leggere gli esempi attentamente. Possono contenere
	dei requisisti non ovvi nella descrizione dell'esercizio. Se qualcosa ti sembra
	 ambiguo, vuol dire che non comprendi il \texttt{C++} abbastanza.
        

	\item Siccome sei autorizzato ad utilizzare gli strumenti del \texttt{C++} 
	dei quali hai appreso all'inizio, non sei autorizzato ad utilizzare nessuna
	libreria esterna. Prima che tu chieda, nelle librerie esterne sono
	compresi anche \texttt{C++11} e derivati, \texttt{Boost} e qualsiasi cosa il 
	tuo amico super skillato ti ha detto che \texttt{C++} non può esistere senza.

	\item Potrebbe esserti richiesto di consegnare un numero importante d classi.
	Può sembrare tedioso, a meno che tu non sia capace di creare uno script
	per il tuo editor di testo preferito.

        \item Leggi ogni esercizio COMPLETAMENTE prima di iniziarlo!
        Seriamente, fallo.
        
        \item Il compilatore da utilizzare è \texttt{clang++}.

        \item Dovrai compilare utilizzare le seguenti flag: \texttt{-Wall -Wextra -Werror}.

	\item Ognuno dei tuoi include deve poter essere incluso indipendentemente
	dagli altri. Chiaramente gli include devono contenere ogni include al quale dipendono.

	 \item Se te lo stessi chiedendo, no non vi è uno stile di scrittura obbligatorio in \texttt{C++}. 
	 Puoi usare lo stile che preferisci, niente restrizioni. Ricorda però che un codice
	 che il tuo peer non può leggere, è un codice che non può valutare.
       
       	    \item Cose importanti: a meno che non sia specificato diversamente, 
	    NON sarai valutato da un programma. Ti è quindi concesso un certo 
	    grado di libertà riguardo all'approccio all'esercizio.
	    Tieni comunque a mente delle costrizioni di ogni esercizio, e NON fare il
	    pigro, perderesti un SACCO di quello che hanno da offrire!
	    

		\item Qualche file estraneo tra quelli che consegnerai non sarà un problema.
	
		\item Anche se il subject dell'esercizio è corto, vale la pena spenderci
		un pò di tempo per essere totalmente certi di aver compreso cosa ci si aspetta
		da te, e che tu l'abbia fatto nella migliore maniera possibile.

        \item Per Odin, per Thor! Usa il tuo cervello!!!

\end{itemize}

%******************************************************************************%
\newpage
