%******************************************************************************%
%                                                                              %
%                        Common Instructions                                   %
%                          for C Projects                                      %
%                                                                              %
%******************************************************************************%

\chapter{Common Instructions}
    \begin{itemize}

      \item 프로젝트는 Norm에 따라 작성되어야 합니다. 만약 추가 파일 혹은 기능이 있다면 이도 Norm을 따라야 하며 Norm 오류가 있으면 \texttt{0점}을 받습니다.

      \item 정의되지 않은 동작과는 별개로 함수가 예기치 않게 종료 (Segmentation fault, bus error, double free, 등)되어서는 안됩니다. 이 경우 프로젝트가 기능하지 않는 것으로 간주되며 평가시 \texttt{0점}을 받습니다.

      \item 필요하다면 모든 힙 할당 메모리 공간을 적절하게 해제해야합니다. 메모리 누출은 용납되지 않습니다.

      \item 프로젝트가 요구한다면,  \texttt{-Wall}, \texttt{-Wextra} 및 \texttt{-Werror} 플래그를 사용하여 소스 파일을 요구된 형식으로 컴파일하는 Makefile을 제출해야하며, Makefile은 relink해서는 안됩니다.

      \item \texttt{Makefile}은 최소한 다음의 규칙을 포함해야 합니다. \texttt{\$(NAME)}, \texttt{all}, \texttt{clean}, \texttt{fclean} and \texttt{re}.

      \item 프로젝트 보너스를 제출하려면 Makefile에 \texttt{bonus} 관련 규칙을 추가해야합니다. 본 프로젝트에서는 금지된 다양한 헤더, 라이브러리 또는 함수를 포함합니다. 보너스는 다른 파일에 있어야합니다, \texttt{\*\_bonus.\{c/h\}}. 필수 항목과 보너스 항목의 평가는 별개로 진행됩니다.

      \item \texttt{libft}를 사용할 수 있는 프로젝트의 경우, 해당 소스 및 관련 \texttt{Makefile}을 \texttt{libft} 폴더에 복사하십시오. 프로젝트의 \texttt{Makefile}은 라이브러리의 \texttt{Makefile}을 사용하여 라이브러리를 컴파일 한 후 프로젝트를 컴파일해야합니다

      \item \textbf{제출할 필요도 없고 채점 되지도 않지만,} 테스트 프로그램을 작성할 것을 추천합니다. 테스트 프로그램은 프로젝트 코드를 쉽게 테스트 할 수 있게 하며  이러한 테스트는 평가하는 동안 굉장히 유용합니다. 실제로, 평가받는 동안 자신이 만든 테스트 및/또는 평가받는 동료의 테스를 자유롭게 사용할 수 있습니다.

      \item 프로젝트를 지정된 Git 저장소에 제출하십시오. Git 저장소에 제출된 작업물만 채점됩니다. Deepthought이 프로젝트를 채점하는 경우, 동료 평가 후에 수행됩니다. Deepthought의 채점 도중에 프로젝트의 어떤 부분에서던 오류가 발생하면 평가가 바로 중지됩니다.
      
    \end{itemize}


