%******************************************************************************%
%                                                                              %
%                        Common Instructions                                   %
%                          for C Projects                                      %
%                                                                              %
%******************************************************************************%

\chapter{Instrucciones generales}
\begin{itemize}
  \item Tu proyecto debe estar escrito siguiendo la Norma. Si tienes archivos o
    funciones adicionales, estas están incluidas en la verificación de la Norma
    y tendrás un \texttt{0} si hay algún error de norma dentro.

  \item Tus funciones no deben terminar de forma inesperada (segfault, bus
    error, double free, etc) ni teener comportamientos indefinidos. Si esto pasa
    tu proyecto será considerado no funcional y recibirás un \texttt{0} durante
    la evaluación.

  \item Toda la memoria alocada en heap deberá liberarse adecuadamente cuando
    sea necesario. No se permitirán leaks de memoria.

  \item Si el subject lo requiere, deberás entregar un \texttt{Makefile} que
    compilará tus archivos fuente al output requerido con las flags
    \texttt{-Wall}, \texttt{-Werror} y \texttt{-Wextra}, por supuesto tu
    \texttt{Makefile} no debe hacer relink.

  \item Tu \texttt{Makefile} debe contener al menos las normas \texttt{\$(NAME)},
    \texttt{all}, \texttt{clean}, \texttt{fclean} y \texttt{re}.

  \item Para entregar los bonus de tu proyecto, deberás incluir una regla
    \texttt{bonus} en tu \texttt{Makefile}, en la que añadirás todos los
    headers, librerías o funciones que estén prohibidas en la parte principal
    del proyecto. Los bonus deben estar en archivos distintos \texttt{\*\_bonus.\{c/h\}}.
    La parte obligatoria y los bonus se evalúan por separado.

  \item Si tu proyecto permite el uso de la \texttt{libft}, deberás copiar su
    fuente y sus \texttt{Makefile} asociados en un directorio \texttt{libft}
    con su correspondiente \texttt{Makefile}. El \texttt{Makefile} de tu
    proyecto debe compilar primero la librería utilizando su \texttt{Makefile},
    y después compilar el proyecto.

  \item Te recomendamos crear programas de prueba para tu proyecto, aunque este
    trabajo \textbf{no será entregado ni evaluado}. Te dará la oportunidad de
    verificar que tu programa funciona correctamente durante tu evaluación y
    la de otros compañeros. Y sí, tienes permitido utilizar estas pruebas
    durante tu evaluación o la de otros compañeros.

  \item Entrega tu trabajo a tu repositorio \texttt{Git} asignado. Solo el
    trabajo de tu repositorio \texttt{Git} será evaluado. Si Deepthought evalúa
    tu trabajo, lo hará después de tus compañeros.
    Si se encuentra un error durante la evaluación de Deepthought, la evalaución
    terminará.
\end{itemize}
