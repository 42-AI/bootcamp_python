%******************************************************************************%
%                                                                              %
%                        Common Instructions                                   %
%                       for Piscine C Projects                                 %
%                                                                              %
%******************************************************************************%


\item Only this page will serve as a reference: do not trust rumors.
\item Watch out! This document could potentially change before submission.
\item Make sure you have the appropriate permissions on your files and directories.
\item You have to follow the \underline{submission procedures} for all your
	exercises.
\item You \underline{cannot} leave \underline{any} additional file in your
	directory than those specified in the subject.
\item Your exercises will be checked and graded by your fellow classmates.
\item On top of that, your exercises will be checked and graded by a program
	called Moulinette.
\item Moulinette is very meticulous and strict in its evaluation of your
	work. It is entirely automated and there is no way to negotiate with it.
	So if you want to avoid bad surprises, be as thorough as possible.
\item Moulinette is not very open-minded. It won’t try and understand your code if it doesn’t respect the Norm.
	Moulinette relies on a program called \texttt{norminette} to check if your files respect the norm.
	TL;DR: it would be idiotic to submit a piece of work that doesn’t pass \texttt{norminette}’s check.
\item Moulinette compiles with these flags: -Wall -Wextra -Werror, and uses \texttt{cc}.
\item If your program doesn’t compile, you’ll get \texttt{0}.
\item These exercises are carefully laid out by order of difficulty -
	from easiest to hardest. We \texttt{will not} take into account a successfully
	completed harder exercise if an easier one is not perfectly functional.
\item Using a forbidden function is considered cheating. Cheaters get \texttt{-42}, and this grade is non-negotiable.
\item You’ll only have to submit a main() function if we ask for a \underline{program}.
\item Got a question? Ask your peer on the right. Otherwise, try your peer
	on the left.
\item Your reference guide is called \texttt{Google / man / the Internet / ...}.
\item Check out the "C Piscine" part of the forum on the intranet, or the slack Piscine.
\item Examine the examples thoroughly. They could very well call for details that are not explicitly mentioned in the subject...
\item By Odin, by Thor! Use your brain!!!