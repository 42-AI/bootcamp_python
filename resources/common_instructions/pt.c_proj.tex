%******************************************************************************%
%                                                                              %
%                        Common Instructions                                   %
%                          for C Projects                                      %
%                                                                              %
%******************************************************************************%

\chapter{Regras gerais}
    \begin{itemize}
      \item O seu projeto deve estar codificado dentro da Norma. Se você possui arquivos ou
        funções bônus, elas serão incluídas na verificação da norma
        e você receberá 0 no projeto se não seguir a norma.

      \itam Suas funções não devem parar inesperadamente (falha
      de segmentação, erro bus, double free, etc.), exceto no caso de um comportamento
      indefinido. Se isso acontecer, o seu projeto será considerado não funcional e você 
      receberá um 0 no projeto.

      \item Qualquer memória alocada no heap deve ser liberada quando necessário.
        Nenhum leak será tolerado.

      \item Se o projeto pedir, você deve fazer um Makefile que compilará as suas 
        fontes para criar a saída solicitada, utilizando as sinalizações \texttt{-Wall},
        \texttt{-Wextra} et \texttt{-Werror}. O seu Makefile não deve ter relink.

      \item Se o seu projeto pedir um Makefile, o seu Makefile deve no mínimo
        conter as regras \texttt(NAME), \texttt{all}, \texttt{clean},
        \texttt{fclean} et \texttt{re}.

      \item Para entregar o bônus, você deve incluir uma regra \texttt{bonus} no seu
        Makefile que vai adicionar os diversos headers, bibliotecas e funções que não são
        autorizadas na parte principal do projeto. Os bônus devem ficar em
        um arquivo \texttt{\*\_bonus.\{c/h\}}. A avaliação da parte obrigatória e
        e da parte bônus são feitas separadamente.

      \item Se o projeto autorizar o seu \texttt{libft}, você deve copiar suas fontes e 
        o seu Makefile associado em uma pasta libfit na raiz.
        O Makefile do seu projeto deve compilar a biblioteca usando o seu Makefile,
        depois compilar o projeto.

      \item Nós recomendamos criar programas de teste para o seu projeto,
        mesmo que esse trabalho \textbf{não seja entregue nem avaliado}. Isso te dará uma chance
        de testar facilmente o seu trabalho assim como o dos seus colegas. 

      \item Você deve entregar o seu trabalho no git que lhe foi atribuído. Somente o trabalho
        colocado no git será avaliado. Se o Deepthought precisar corrigir o seu trabalho, isso será feito
        no fim do processo das avaliações dos colegas.
        Se um erro acontecer durante a avaliação Deepthought, ela será finalizada.
    \end{itemize}
