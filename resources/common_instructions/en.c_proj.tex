%******************************************************************************%
%                                                                              %
%                        Common Instructions                                   %
%                          for C Projects                                      %
%                                                                              %
%******************************************************************************%

\chapter{Common Instructions}
    \begin{itemize}

      \item Your project must be written in C.

      \item Your project must be written in accordance with the Norm.
        If you have bonus files/functions, they are included in the norm check
        and you will receive a \texttt{0} if there is a norm error inside.

      \item Your functions should not quit unexpectedly (segmentation
        fault, bus error, double free, etc) apart from undefined
        behaviors. If this happens, your project will be considered non
        functional and will receive a \texttt{0} during the evaluation.

      \item All heap allocated memory space must be properly freed
        when necessary. No leaks will be tolerated.

      \item If the subject requires it, you must submit a \texttt{Makefile}
        which will compile your source files to the required output
        with the flags \texttt{-Wall}, \texttt{-Wextra} and \texttt{-Werror},
        use cc, and your Makefile must not relink.

      \item Your \texttt{Makefile} must at least contain the rules
        \texttt{\$(NAME)}, \texttt{all}, \texttt{clean},
        \texttt{fclean} and \texttt{re}.

      \item To turn in bonuses to your project, you must include a rule
        \texttt{bonus} to your Makefile, which will add all the various headers, 
        librairies or functions that are forbidden on the main part of the project. 
        Bonuses must be in a different file \texttt{\*\_bonus.\{c/h\}} if the subject does not specify anything else.
	Mandatory and bonus part evaluation is done separately.

      \item If your project allows you to use your \texttt{libft}, you must copy its sources
        and its associated \texttt{Makefile} in a \texttt{libft} folder with its associated
        Makefile. Your project's \texttt{Makefile} must compile the library by using its
        \texttt{Makefile}, then compile the project.

      \item We encourage you to create test programs for your
        project even though this work \textbf{won't have to be
          submitted and won't be graded}. It will give you a chance
        to easily test your work and your peers’ work. You will find
        those tests especially useful during your defence. Indeed,
        during defence, you are free to use your tests and/or the
        tests of the peer you are evaluating.

      \item Submit your work to your assigned git repository. Only the work in the
        git repository will be graded. If Deepthought is assigned to grade your
        work, it will be done after your peer-evaluations.
        If an error happens in any section of your work during Deepthought's grading,
        the evaluation will stop.
    \end{itemize}
