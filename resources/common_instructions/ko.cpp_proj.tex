%******************************************************************************%
%                                                                              %
%                  instructions.tex for 42's Piscine C++                       %
%                  Created on : Mon Sep  8 15:57:19 2014                       %
%                  Made by : David "Thor" GIRON <thor@42.fr>                   %
%                                                                              %
%******************************************************************************%

\chapter{General rules}


    \begin{itemize}

 		\item 해더에 함수를 구현하거나, 보호 되지 않은 해더가 있을경우 과제는 0점 처리됩니다.

 		\item 모든 출력은 표준 출력으로 가고 달리 지정하지 않는 한 개행으로 끝납니다.

 		\item 부과 된 파일 이름 뒤에 문자, 클래스 이름, 함수 이름 및 메소드 이름이 와야합니다.

 		\item 잊지마세요 : 이제  \texttt{C}가 아니라 \texttt{C++}로 코딩을 하게 됩니다. 따라서:
		  
		  \begin{itemize}
		  
		  \item 다음 함수들은 사용이 금지되며, 사용 시 \texttt{0점} 처리됩니다. 예외는 없습니다. : \texttt{* alloc}, \texttt{* printf} , \texttt{free}.

		  \item 기본적으로 표준 라이브러리의 모든 것은 사용할 수 있습니다. 그러나 새로운 언어를 배우는 기회이니만큼 C에서 사용하던 함수를 C ++ 스러운 함수로 바꿔 사용해보십시오. 그리고 안됩니다, 실제로 STL을 사용할 수 있을 때까지 (즉, 모듈 08까지) STL을 사용할 수 없습니다. 즉, vectors/lists/maps/등 ... 또는 <algorithm> 라이브러리를 필요로 하는 모든 것은 쓸수없습니다.

		  \end{itemize}

		\item 실제 명시적으로 금지 된 함수를  사용할 시, \texttt{0점} 처리됩니다. 예외는 없습니다.

        \item 달리 명시되지 않는 한 \texttt{C++} 키워드 \texttt{"using namespace"} 및 \texttt{"friend"}는 금지됩니다. 사용 시 \texttt {-42점} 처리됩니다.

        \item달리 명시되지 않는 한 클래스와 관련된 파일은 항상 \texttt{ClassName.hpp}  및\texttt{ClassName.cpp}
   로 만듭니다.

        \item 제출 해야 할 디렉토리는 \texttt{ex00/}, \texttt{ex01/}, \dots, \texttt{exn/} 입니다.

        \item 예제를 철저히 읽어야합니다. 예제에는 설명에 명확하지 않은 요구 사항이 포함될 수 있습니다. 뭔가 애매모호하면 \texttt{C++}를 충분히 이해하지 못한것입니다.

        \item 처음부터  \texttt{C++} 도구를 사용할 수 있으므로, 외부 라이브러리는 금지됩니다. 아마 물어볼거같아서 미리 답하자면  \texttt{C++11} 이나 다른 파생된 언어들,  \texttt{Boost} 또는 킹왕짱 친구가 알려준, 이것들 없이는 \texttt{C++}가 존재할 수 없다는, 이것들은 모두 사용하실수 없습니다.

        \item 많은 양의 중요한 클레스들을 제출하라고 요구받을 수 있습니다. 본인의 최애 텍스트 편집기를 설정하는 방법을 알지 않는한 지루하게 느껴질수도 있습니다.

        \item 시작하기 전에 각 과제를 정확히 읽어야합니다! 꼭이요!

        \item 사용할 컴파일러는 \texttt{clang++}입니다.

        \item 코드는 다음과 같이 컴파일되어야합니다
          플래그 : \texttt{-Wall -Wextra -Werror}.

        \item includes는 각각 독립적으로  포함될수 있어야합니다. 당연하지만, includes는 의존하고 있는 모든 includes들을 포함해야합니다.

        \item 혹시나 궁금할까봐 미리 말해두는데, \texttt{C++}에서는 특정한 코딩 스타일을 강요하지 않습니다. 제한 없이 원하시는 스타일로 즐기시길바랍니다. 그러나 동료 평가자가 읽을 수 없는 코드는 채점 할 수없는 코드입니다.

 	   \item 지금 중요한 것들 : 달리 명시 적으로 언급되지 않는 한, 프로그램에 의해 채점되지 않을 것입니다. 그러므로 당신은 여유가 있습니다. 따라서 과제를 어떻게 하느냐에 대한 방법에 어느 정도의 자유가 주어집니다. 그러나 각 과제의 제약 조건을 염두에 두시고! 절대 게을러지지 마세요! 그들이 제공하는 많은 것들을 놓치게 될 것입니다!

		\item 요구한 파일보다 더 많은 파일로 코드를 분리하여 제출해도 괜찮습니다. 프로그램이 채점을 하지 않는 이상 자유롭게 쓰셔도 됩니다.

		\item 과제가 짧더라도 시간을 두고 완전히 이해하고 최선의 방법으로 풀기를 바랍니다.


        \item By 오딘 (Odin), By 토르 (Thor)! 머리를 써라!!!

\end{itemize}

%******************************************************************************%
\newpage
