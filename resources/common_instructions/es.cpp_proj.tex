%******************************************************************************%
%                                                                              %
%                  instructions.tex for 42's Piscine C++                       %
%                  Created on : Mon Sep  8 15:57:19 2014                       %
%                  Made by : David "Thor" GIRON <thor@42.fr>                   %
%                                                                              %
%******************************************************************************%


\chapter{Instrucciones generales}

Para los módulos de C++ utilizarás y aprenderás exclusivamente C++98.
Tu objetivo es aprender las nociones de la programación orientada a objetos.
Sabemos que las últimas versiones de C++ son muy diferentes en muchos aspectos,
si quieres volverte un experto en C++ deberás aprender C++ moderno más adelante.
Este es el principio de tu largo viaje por C++, es cosa tuya ir más allá
después del common core.

\begin{itemize}
  \item Cualquier función implementada en un header (excepto en el caso de
    templates), y cualquier header desprotegido, significa un 0 en el ejercicio.

  \item Todos los output deben ir al standard output, y deben terminar con
    un salto de línea, salvo que se indique lo contrario.

  \item Se deben seguir los nombres de archivos impuestos al pie de la letra,
    así como las clases, funciones y métodos.

  \item Recuerda: estás programando en \texttt{C++}, no en \texttt{C}.
    Por lo tanto:

	\begin{itemize}
    \item Las siguientes funciones están PROHIBIDAS, y su uso será sancionado
      con un \texttt{0}, sin preguntas: \texttt{*alloc}, \texttt{*printf} y
      \texttt{free}.

    \item Tienes permitido utilizar básicamente todo de la librería estándar.
      SIN EMBARGO, sería inteligente utilizar la versión de C++ de las
      funciones a las que estás acostumbrado en C, en lugar de simplemente
      seguir utilizando lo que sabes... En realidad, estás aprendiendo un
      lenguaje nuevo. Y NO, no tienes permitido utilizar STL hasta que debas
      hacerlo (es decir, el módulo 08). Esto significa que nada de vectors/
      lists/maps/etc. O nada que requiera un ``include <algorithm>''
      hasta entonces.
  \end{itemize}

  \item De hecho, el uso de cualquier función o mecánica explícitamente
    prohibida será recompensada con un \texttt{0}, sin preguntas.

  \item Ten en cuenta que, salvo especificado de otro modo, las palabras de
    \texttt{C++} \texttt{"using namespace"} y \texttt{"friend"} están
    terminantemente prohibidas. Su uso será sancionado con \texttt{-42},
    sin preguntas.

  \item Los archivos asociados con una clase se llamarán siempre
    \texttt{ClassName.hpp} y \texttt{ClassName.cpp}, salvo especificado
    de otro modo.

  \item Entrega en directorios denominados \texttt{ex00/}, \texttt{ex01/}
    \ldots\texttt{exn/}.

  \item Debes leer los ejemplos con cuidado. Pueden contener requisitos no
    tan obvios en la descripción del ejercicio.

  \item Dado que tienes permitido utilizar herramientas de \texttt{C++}
    que llevas aprendiendo desde el principio, no tienes permitido utilizar
    librerías externas. Y antes de que te lo preguntes, esto significa
    que ningún derivado de \texttt{C++11}, ni \texttt{Boost} o similares
    se permite.

  \item Puede que se requiera entregar un importante número de clases. Esto
    puede parecer tedioso, salvo que sepas instalar scripts en tu editor
    de texto favorito.

  \item Lee cuidadosamente los ejercicios POR COMPLETO antes de empezarlos.
    En serio, hazlo.

  \item El compilador a usar es \texttt{clang++}.

  \item Tu código debe compilar con las flags: \texttt{-Wall -Werror -Wextra}.

  \item Cada uno de tus \texttt{include} debe poder incluirse
    independientemente del resto. Los \texttt{include} deben contener
    obviamente los \texttt{include} de los que dependan.

  \item Por si te lo preguntas, no se requiere ningún estilo de código durante
    \texttt{C++}. Puedes utilizar una guía de estilos que te guste, sin
    limitaciones. Recuerda que si tu evaluador no es capaz de leer tu código,
    tampoco lo será de evaluarte.

  \item Algo importante: NO te evaluará un programa, salvo que el subject
    lo indique explícitamente. Por lo tanto, tienes cierta libertad en cómo
    hagas los ejercicios. Sin embargo, sé inteligente con los principios
    de cada ejercicio, y NO seas perezoso, te perderás MUCHO de lo que
    estos proyectos te pueden ofrecer.

  \item No es un problema real si tienes archivos adicionales a los que
    se te solicita, puedes elegir separar el código en más archivos de los
    que se te piden. Siéntete libre, siempre y cuando el resultado
    no lo evalúe un programa.

  \item Aunque el subject de un ejercicio sea corto, merece la pena gastar
    algo de tiempo para estar absolutamente seguro de que entiendes lo que
    se espera que entiendas, y que lo has hecho de la mejor forma posible.

  \item Por Odin, por Thor. Utiliza tu cerebro.

\end{itemize}

%******************************************************************************%
\newpage
