\chapter{Exercise 02}
\extitle{ScrapBooker}
\turnindir{ex02}
\exnumber{02}
\exfiles{ScrapBooker.py}
\exforbidden{None}
\makeheaderfilesforbidden

% ================================= %
\section*{Objective}
% --------------------------------- %
Manipulation and initiation to slicing on numpy arrays.

% ================================= %
\section*{Instructions}
% --------------------------------- %
Implement a class named \texttt{ScrapBooker} with the following methods:
\begin{itemize}
  \item \texttt{crop},
  \item \texttt{thin},
  \item \texttt{juxtapose},
  \item \texttt{mosaic}.
\end{itemize}

\begin{minted}[bgcolor=darcula-back,formatcom=\color{lightgrey},fontsize=\scriptsize]{python}
# within the class
def crop(self, array, dim, position=(0,0)):
    """
    Crops the image as a rectangle via dim arguments (being the new height
    and width of the image) from the coordinates given by position arguments.
    
    Args:
    -----
      array:    numpy.ndarray
      dim:      tuple of 2 integers.
      position: tuple of 2 integers.

    Returns:
    -------
      new_arr:  the cropped numpy.ndarray.
      None:     (if the combination of parameters is not possible).
    
    Raises:
    ------
      This function should not raise any Exception.
    """
    ... your code ...

def thin(self, array, n, axis):
    """
    Deletes every n-th line pixels along the specified axis (0: vertical, 1: horizontal)

    Args:
    -----
      array:   numpy.ndarray.
      n:       non null positive integer lower than the number of row/column of the array
                (depending of axis value).
      axis:    positive non null integer.

    Returns:
    -------
      new_arr:  thined numpy.ndarray.
      None:     (if the combination of parameters is not possible).

    Raises:
    ------
      This function should not raise any Exception.
    """
    ... your code ...

def juxtapose(self, array, n, axis):
    """
    Juxtaposes n copies of the image along the specified axis.
 
    Args:
    -----
      array:    numpy.ndarray.
      n:        positive non null integer.
      axis:     integer of value 0 or 1.
 
    Returns:
    -------
      new_arr:  juxtaposed numpy.ndarray.
      None:     (if the combination of parameters is not possible).
 
    Raises:
    -------
      This function should not raise any Exception.
    """
    ... your code ...
\end{minted}
\newpage
\begin{minted}[bgcolor=darcula-back,formatcom=\color{lightgrey},fontsize=\scriptsize]{python}
def mosaic(self, array, dim):
    """
    Makes a grid with multiple copies of the array. The dim argument specifies
    the number of repetition along each dimensions.

    Args:
    -----
      array:    numpy.ndarray.
      dim:      tuple of 2 integers.
    
    Return:
    -------
      new_arr:  mosaic numpy.ndarray.
      None      (combinaison of parameters not compatible).
    
    Raises:
    -------
      This function should not raise any Exception.
      """
      ... your code ...
\end{minted}
\newline
In this exercise, when specifying positions or dimensions, we will assume
that the first coordinate is counted along the vertical axis starting from
the top, and that the second coordinate is counted along the horizontal axis
starting from the left. Indexing starts from 0.\\

\parbox{\textwidth}{
  e.g.:\\
  (1,3)\\
  .....\\
  ...x.\\
  .....}


% ================================= %
\section*{Examples}
% --------------------------------- %
\begin{minted}[bgcolor=darcula-back,formatcom=\color{lightgrey},fontsize=\scriptsize]{python}
import numpy as np
from ScrapBooker import ScrapBooker

spb = ScrapBooker()
arr1 = np.arange(0,25).reshape(5,5)
spb.crop(arr1, (3,1),(1,0))
#Output :
array([[ 5],
       [10],
       [15]])


arr2 = np.array("A B C D E F G H I".split() * 6).reshape(-1,9)
spb.thin(arr2,3,0)
#Output :
array([['A', 'B', 'D', 'E', 'G', 'H', 'J', 'K'],
       ['A', 'B', 'D', 'E', 'G', 'H', 'J', 'K'],
       ['A', 'B', 'D', 'E', 'G', 'H', 'J', 'K'],
       ['A', 'B', 'D', 'E', 'G', 'H', 'J', 'K'],
       ['A', 'B', 'D', 'E', 'G', 'H', 'J', 'K'],
       ['A', 'B', 'D', 'E', 'G', 'H', 'J', 'K']], dtype='<U1')


arr3 = np.array([[1, 2, 3],[1, 2, 3],[1, 2, 3]])
spb.juxtapose(arr3, 3, 1)
#Output :
array([[1, 2, 3, 1, 2, 3, 1, 2, 3],
       [1, 2, 3, 1, 2, 3, 1, 2, 3],
       [1, 2, 3, 1, 2, 3, 1, 2, 3]])
\end{minted}