\chapter{Exercise 00}
\extitle{NumPyCreator}
\turnindir{ex00}
\exnumber{00}
\exfiles{NumPyCreator.py}
\exforbidden{None}
\makeheaderfilesforbidden

% ================================== %
\section*{Objective}
% ---------------------------------- %
Introduction to the Numpy library.

% ================================== %
\section*{Instructions}
% ---------------------------------- %
Write a class named \texttt{NumPyCreator}, that implements all of the following methods.\\
\\
Each method receives as an argument a different type of data structure and transforms it into a Numpy array:
\begin{itemize}
  \item \texttt{from\_list(self, lst)}: takes a list or nested lists and returns its corresponding Numpy array.
  \item \texttt{from\_tuple(self, tpl)}: takes a tuple or nested tuples and returns its corresponding Numpy array.
  \item \texttt{from\_iterable(self, itr)}: takes an iterable and returns an array which contains all of its elements.
  \item \texttt{from\_shape(self, shape, value)}: returns an array filled with the same value.
  The first argument is a tuple which specifies the shape of the array, and the second argument specifies the value of the elements. 
  This value must be 0 by default.
  \item \texttt{random(self, shape)}: returns an array filled with random values.
  It takes as an argument a tuple which specifies the shape of the array.
  \item \texttt{identity(self, n)}: returns an array representing the identity matrix of size n.
\end{itemize}
\textit{\textbf{BONUS:}} Add to these methods an optional argument which specifies the datatype (dtype) of the array (e.g. to represent its elements as integers, floats, ...)

\hint{Each of these methods can be implemented in one line. You only need to find the right Numpy functions.}

% ================================== %
\section*{Examples}
% ---------------------------------- %
\begin{minted}[bgcolor=darcula-back,formatcom=\color{lightgrey},fontsize=\scriptsize]{python}
from NumpyCreator import NumpyCreator
npc = NumpyCreator()

npc.from_list([[1,2,3],[6,3,4]])
# Output :
array([[1, 2, 3],
       [6, 3, 4]])


npc.from_list([[1,2,3],[6,4]])
# Output :
None


npc.from_list([[1,2,3],['a','b','c'],[6,4,7]])
# Output :
array([['1','2','3'],
       ['a','b','c'],
       ['6','4','7'], dtype='<U21'])


npc.from_list(((1,2),(3,4)))
# Output :
None


npc.from_tuple(("a", "b", "c"))
# Output :
array(['a', 'b', 'c'])


npc.from_tuple(["a", "b", "c"])
# Output :
None


npc.from_iterable(range(5))
# Output :
array([0, 1, 2, 3, 4])


shape=(3,5)
npc.from_shape(shape)
# Output :
array([[0, 0, 0, 0, 0],
       [0, 0, 0, 0, 0],
       [0, 0, 0, 0, 0]])


npc.random(shape)
# Output :
array([[0.57055863, 0.23519999, 0.56209311, 0.79231567, 0.213768 ],
      [0.39608366, 0.18632147, 0.80054602, 0.44905766, 0.81313615],
      [0.79585328, 0.00660962, 0.92910958, 0.9905421 , 0.05244791]])


npc.identity(4)
# Output :
array([[1., 0., 0., 0.],
       [0., 1., 0., 0.],
       [0., 0., 1., 0.],
       [0., 0., 0., 1.]])
\end{minted}