\chapter{Exercise 04}
\extitle{K-means Clustering}
\turnindir{ex04}
\exnumber{04}
\exfiles{Kmeans.py}
\exforbidden{Any functions allowing you to perform K-Means}
\makeheaderfilesforbidden

ALERT! DATA CORRUPTED

% ================================= %
\section*{Objective}
% --------------------------------- %
Implementation of a basic K-means algorithm.

% ================================= %
\section*{Instructions}
% --------------------------------- %
The solar system census dataset is corrupted! The citizens' homelands are missing!\\  
\\
You must implement the K-means clustering algorithm in order to recover the citizens' origins.\\
\\
You can find good explanations on how K-means is working here:
\href{https://en.wikipedia.org/wiki/K-means_clustering}{Wikipedia - K-Means clustering}\\
\\
The missing part is how to compute the distance between 2 data points (cluster centroid or a row in the data).\\
\\
In our case, the data we have to process is composed of 3 values (height, weight and bone\_density).\\
\\
Thus, each data point is a vector of 3 values.\\
\\
Now that we have mathematically defined our data points (vector of 3 values), it is very easy to compute the distance between two points using vector properties.\\
You can use L1 distance, L2 distance, cosine similarity, and so forth...\\
Choosing the distance to use is called hyperparameter tuning.\\
\newline
I would suggest you to try with the easiest setting (L1 distance) first.\\
\\
What you will notice is that the final result of the "training"/"fitting" will depend a lot on the random initialization.\\
\\
Commonly, in machine-learning libraries, K-means is run multiple times (with different random initializations) and the best result is saved.\\  
\\
NB: To implement the fit function, keep in mind that a centroid can be considered as the gravity center of a set of points.\\
\\
Your program \texttt{Kmeans.py} takes 3 parameters: \texttt{filepath}, \texttt{max\_iter} and \texttt{ncentroid}:

\begin{42console}
  python Kmeans.py filepath='../ressources/solar_system_census.csv' ncentroid=4 max_iter=30
\end{42console}

Your program is expected to:
\begin{itemize}
  \item parse its arguments,
  \item read the dataset,
  \item fit the dataset,
  \item display the coordinates of the different centroids and the associated region (for the case \texttt{ncentroid=4}),
  \item display the number of individuals associated to each centroid,
  \item (Optional) display the results  on 3 differents plots, corresponding to 3 combinations of 2 parameters, using different colors to distinguish between Venus,Earth, Mars and Belt asteroids citizens.
\end{itemize}

Create the class \texttt{KmeansClustering} with the following methods:  

\begin{minted}[bgcolor=darcula-back,formatcom=\color{lightgrey},fontsize=\scriptsize]{python}
class KmeansClustering:
    def __init__(self, max_iter=20, ncentroid=5):
        self.ncentroid = ncentroid # number of centroids
        self.max_iter = max_iter # number of max iterations to update the centroids
        self.centroids = [] # values of the centroids
        
    def fit(self, X):
        """
        Run the K-means clustering algorithm.
        For the location of the initial centroids, randomly pick n centroids from the dataset.
        Args:
        -----
          X: has to be an numpy.ndarray, a matrice of dimension m * n.
        Return:
        -------
          None.
        Raises:
        -------
          This function should not raise any Exception.
        """
        ... your code ...

    def predict(self, X):
        """
        Predict from wich cluster each datapoint belongs to.
        Args:
        -----
          X: has to be an numpy.ndarray, a matrice of dimension m * n.
        Return:
        -------
          the prediction has a numpy.ndarray, a vector of dimension m * 1.
        Raises:
        -------
          This function should not raise any Exception.
        """
        ... your code ...
\end{minted}

% ================================= %
\section*{Dataset}
% --------------------------------- %
The dataset, named \textbf{solar\_system\_census} can be found in
the resources folder.\\
\\
It is a part of the solar system census dataset, and contains biometric
data such as the height, weight, and bone density of solar
system citizens.\\
\\
Solar citizens come from four registered areas:
\begin{itemize}
  \item The flying cities of Venus,
  \item United Nations of Earth,
  \item Mars Republic,
  \item Asteroids' Belt colonies.  
\end{itemize}
Unfortunately the data about the planets of origin was lost...\\
Use your K-means algorithm to recover it!\\
Once your clusters are found, try to find matches between clusters and the citizens' homelands.\\

\hint{
  \begin{itemize}
    \item People are slender on Venus than on Earth.  
    \item People of the Martian Republic are taller than on Earth.  
    \item Citizens of the Belt are the tallest of the solar system and have the lowest bone density due to the lack of gravity.  
  \end{itemize}
}

% ================================= %
\section*{Examples}
% --------------------------------- %
Here is an example of the K-means algorithm in action:\\
\url{https://i.ibb.co/bKFVVx2/ezgif-com-gif-maker.gif}