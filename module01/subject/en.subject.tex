% vim: set ts=4 sw=4 tw=80 noexpandtab:

\documentclass{42-en}
\usepackage{amsmath}

%******************************************************************************%
%                                                                              %
%                                   Prologue                                   %
%                                                                              %
%******************************************************************************%
\usepackage[
    type={CC},
    modifier={by-nc-sa},
    version={4.0},
]{doclicense}

%****************************************************************%
%                  Re/definition of commands                     %
%****************************************************************%

\newcommand{\ailogo}[1]{\def \@ailogo {#1}}\ailogo{assets/42ai_logo.pdf}

%%  Redefine \maketitle
\makeatletter
\def \maketitle {
  \begin{titlepage}
    \begin{center}
	%\begin{figure}[t]
	  %\includegraphics[height=8cm]{\@ailogo}
	  \includegraphics[height=8cm]{assets/42ai_logo.pdf}
	%\end{figure}
      \vskip 5em
      {\huge \@title}
      \vskip 2em
      {\LARGE \@subtitle}
      \vskip 4em
    \end{center}
    %\begin{center}
	  %\@author
    %\end{center}
	%\vskip 5em
  \vfill
  \begin{center}
    \emph{\summarytitle : \@summary}
  \end{center}
  \vspace{2cm}
  %\vskip 5em
  %\doclicenseThis
  \end{titlepage}
}
\makeatother

\makeatletter
\def \makeheaderfilesforbidden
{
  \noindent
  \begin{tabularx}{\textwidth}{|X X  X X|}
    \hline
  \multicolumn{1}{|>{\raggedright}m{1cm}|}
  {\vskip 2mm \includegraphics[height=1cm]{assets/42ai_logo.pdf}} &
  \multicolumn{2}{>{\centering}m{12cm}}{\small Exercise : \@exnumber } &
  \multicolumn{1}{ >{\raggedleft}p{1.5cm}|}
%%              {\scriptsize points : \@exscore} \\ \hline
              {} \\ \hline

  \multicolumn{4}{|>{\centering}m{15cm}|}
              {\small \@extitle} \\ \hline

  \multicolumn{4}{|>{\raggedright}m{15cm}|}
              {\small Turn-in directory : \ttfamily
                $ex\@exnumber/$ }
              \\ \hline
  \multicolumn{4}{|>{\raggedright}m{15cm}|}
              {\small Files to turn in : \ttfamily \@exfiles }
              \\ \hline

  \multicolumn{4}{|>{\raggedright}m{15cm}|}
              {\small Forbidden functions : \ttfamily \@exforbidden }
              \\ \hline

%%  \multicolumn{4}{|>{\raggedright}m{15cm}|}
%%              {\small Remarks : \ttfamily \@exnotes }
%%              \\ \hline
\end{tabularx}
%% \exnotes
\exrules
\exmake
\exauthorize{None}
\exforbidden{None}
\extitle{}
\exnumber{}
}
\makeatother

%%  Syntactic highlights
\makeatletter
\newenvironment{pythoncode}{%
  \VerbatimEnvironment
  \usemintedstyle{emacs}
  \minted@resetoptions
  \setkeys{minted@opt}{bgcolor=black,formatcom=\color{lightgrey},fontsize=\scriptsize}
  \begin{figure}[ht!]
    \centering
    \begin{minipage}{16cm}
      \begin{VerbatimOut}{\jobname.pyg}}
{%[
      \end{VerbatimOut}
      \minted@pygmentize{c}
      \DeleteFile{\jobname.pyg}
    \end{minipage}
\end{figure}}
\makeatother

\usemintedstyle{native}

\begin{document}

% =============================================================================%
%                     =====================================                    %

\title{Python \& ML - Module 01}
\subtitle{Basic 2}
\author{
  Maxime Choulika (cmaxime), Pierre Peigné (ppeigne), Matthieu David (mdavid)
}


\summary
{
	The goal of the module is to get familiar with object-oriented programming
	and much more.
}

\maketitle
%******************************************************************************%
%                                                                              %
%                        Common Instructions                                   %
%                          for Python Projects                                 %
%                                                                              %
%******************************************************************************%

\chapter{Common Instructions}
\begin{itemize}
  \item The version of Python recommended to use is 3.7, you can
  check the version of Python with the following command: \texttt{python -V}
  
  \item The norm: during this piscine, it is recommended to follow the
  \href{https://www.python.org/dev/peps/pep-0008/}{PEP 8 standards}, though it is not mandatory.
  You can install \href{https://pypi.org/project/pycodestyle}{pycodestyle} which
  is a tool to check your Python code.
  \item The function \texttt{eval} is never allowed.
  \item The exercises are ordered from the easiest to the hardest.
  \item Your exercises are going to be evaluated by someone else,
  so make sure that your variable names and function names are appropriate and civil.
  \item Your manual is the internet.
  
  \item You can also ask questions in the \texttt{\#bootcamps} channel in the \href{https://42-ai.slack.com}{42AI}
  or \href{42born2code.slack.com}{42born2code}.
  
  \item If you find any issue or mistake in the subject please create an issue on \href{https://github.com/42-AI/bootcamp_python/issues}{42AI repository on Github}.
  
  \item We encourage you to create test programs for your
  project even though this work \textbf{won't have to be
  submitted and won't be graded}. It will give you a chance
  to easily test your work and your peers’ work. You will find
  those tests especially useful during your defence. Indeed,
  during defence, you are free to use your tests and/or the
  tests of the peer you are evaluating.
  
  \item Submit your work to your assigned git repository. Only the work in the
  git repository will be graded. If Deepthought is assigned to grade your
  work, it will be run after your peer-evaluations.
  If an error happens in any section of your work during Deepthought's grading,
  the evaluation will stop.
\end{itemize}
\newpage
\tableofcontents
\startexercices

%                     =====================================                    %
% =============================================================================%


%******************************************************************************%
%                                                                              %
%                                   Exercises                                  %
%                                                                              %
%******************************************************************************%

% ============================================== %
% ===========================(start ex 00)       %
\chapter{Exercise 00}
\extitle{The Book}
\turnindir{ex00}
\exnumber{00}
\exfiles{book.py, recipe.py, test.py}
\exforbidden{None}
\makeheaderfilesforbidden



% ================================== %
\section*{Objective}
% ---------------------------------- %
The goal of the exercise is to get you familiar with the notions of
classes and the manipulation of the objects related to those classes.

% ================================== %
\section*{Instructions}
% ---------------------------------- %
You will have to make a class \texttt{Book} and a class \texttt{Recipe}.
The classes \texttt{Book} and \texttt{Recipe} will be written in 
\texttt{book.py} and \texttt{recipe.py} respectively.


Let's describe the \texttt{Recipe} class. It has some attributes:
\begin{itemize}
  \item \texttt{name} (str): name of the recipe,
  \item \texttt{cooking\_lvl} (int): range from $1$ to $5$,
  \item \texttt{cooking\_time} (int): in minutes (no negative numbers),
  \item \texttt{ingredients} (list): list of all ingredients each represented by a string,
  \item \texttt{description} (str): description of the recipe,
  \item \texttt{recipe\_type} (str): can be "starter", "lunch" or "dessert".
\end{itemize}

You have to \textbf{initialize} the object \texttt{Recipe} and check all its values, only the description can be empty.
In case of input errors, you should print what they are and exit properly.

You will have to implement the built-in method \texttt{\_\_str\_\_}.
It's the method called when the following code is executed:

\begin{minted}[bgcolor=darcula-back,formatcom=\color{lightgrey},fontsize=\scriptsize]{python}
tourte = Recipe(...)
to_print = str(tourte)
print(to_print)
\end{minted}

It is implemented this way:

\begin{minted}[bgcolor=darcula-back,formatcom=\color{lightgrey},fontsize=\scriptsize]{python}
def __str__(self):
    """Return the string to print with the recipe info"""
    txt = ""
    """Your code here"""
    return txt
\end{minted}

The \texttt{Book} class also has some attributes:
\begin{itemize}
  \item \texttt{name} (str): name of the book,
  \item \texttt{last\_update} \href{https://docs.python.org/3/library/datetime.html}{(datetime)}: the date of the last update,
  \item \texttt{creation\_date} \href{https://docs.python.org/3/library/datetime.html}{(datetime)}: the creation date,
  \item \texttt{recipes\_list} (dict): a dictionnary with 3 keys: "starter", "lunch", "dessert".
\end{itemize}


You will have to implement some methods in \texttt{Book} class:

\begin{minted}[bgcolor=darcula-back	,formatcom=\color{lightgrey},fontsize=\scriptsize]{python}
def get_recipe_by_name(self, name):
    """Prints a recipe with the name \texttt{name} and returns the instance"""
    #... Your code here ...

def get_recipes_by_types(self, recipe_type):
    """Get all recipe names for a given recipe_type """
    #... Your code here ...

def add_recipe(self, recipe):
    """Add a recipe to the book and update last_update"""
    #... Your code here ...
\end{minted}

You have to handle the error if the argument passed in \texttt{add\_recipe} is not a \texttt{Recipe}.


Finally, you will provide a \texttt{test.py} file to test your classes and prove that they are working properly.
You can import all the classes into your \texttt{test.py} file by adding these lines at the top of the \texttt{test.py} file:

\begin{minted}[bgcolor=darcula-back	,formatcom=\color{lightgrey},fontsize=\scriptsize]{python}
from book import Book
from recipe import Recipe

# ... Your tests ...
\end{minted}

% ===========================(fin ex 00)         %
% ============================================== %
\newpage

% ============================================== %
% ===========================(start ex 01)       %
\chapter{Exercise 01}
\extitle{Family tree}
\turnindir{ex01}
\exnumber{01}
\exfiles{game.py}
\exforbidden{None}
\makeheaderfilesforbidden


% ================================= %
\section*{Objective}
% --------------------------------- %
The goal of the exercise is to tackle the notion inheritance of class.

% ================================= %
\section*{Instructions}
% --------------------------------- %
Create a \texttt{GotCharacter} class and initialize it with the following attributes:
\begin{itemize}
  \item \texttt{first\_name},
  \item  \texttt{is\_alive} (by default is \texttt{True}).
\end{itemize}


Pick up a GoT House (e.g., Stark, Lannister...) and create a child class that inherits from \texttt{GotCharacter} and 
define the following attributes:
\begin{itemize}
  \item \texttt{family\_name} (by default should be the same as the Class)
  \item \texttt{house\_words} (e.g., the House words for the Stark House is: "Winter is Coming")
\end{itemize}

\begin{minted}[bgcolor=darcula-back,formatcom=\color{lightgrey},fontsize=\scriptsize]{python}
  class Stark(GotCharacter):
    def __init__(self, first_name=None, is_alive=True):
        super().__init__(first_name=first_name, is_alive=is_alive)
        self.family_name = "Stark"
        self.house_words = "Winter is Coming"
\end{minted}

Add two methods to your child class:
\begin{itemize}
  \item \texttt{print\_house\_words}: prints the House words,
  \item \texttt{die}: changes the value of \texttt{is\_alive} to \texttt{False}.
\end{itemize}

% ================================= %
\section*{Examples}
% --------------------------------- %

Running commands in the Python console, an example of what you should get:

\begin{minted}[bgcolor=darcula-back,formatcom=\color{lightgrey},fontsize=\scriptsize]{python}
	$> python
  >>> from game import Stark
  
  >>> arya = Stark("Arya")
  >>> print(arya.__dict__)
  {'first_name': 'Arya', 'is_alive': True, 'family_name': 'Stark', 'house_words': 'Winter is Coming'}
  
  >>> arya.print_house_words()
  Winter is Coming
  
  >>> print(arya.is_alive)
  True
  
  >>> arya.die()
  >>> print(arya.is_alive)
  False
\end{minted}

You can add any attribute or method you need to your class and format the docstring the way you want to.
Feel free to create other children of \texttt{GotCharacter} class.

\begin{minted}[bgcolor=darcula-back,formatcom=\color{lightgrey},fontsize=\scriptsize]{python}
	$> python
  >>> from game import Stark
  
  >>> arya = Stark("Arya")
  >>> print(arya.__doc__)
  A class representing the Stark family. Or when bad things happen to good people.
\end{minted}


% ===========================(fin ex 01)         %
% ============================================== %

\newpage

% ============================================== %
% ===========================(start ex 02)       %
\chapter{Exercise 02}
\extitle{The Vector}
\turnindir{ex02}
\exnumber{02}
\exfiles{vector.py, test.py}
\exforbidden{Numpy library}
\makeheaderfilesforbidden

% ================================= %
\section*{Objective}
% --------------------------------- %
The goal of the exercise is to get you used with built-in methods, more
particularly with those allowing to perform operations.
Student is expected to code built-in methods for vector-vector and
vector-scalar operations as rigorously as possible.

% ================================= %
\section*{Instructions}
% --------------------------------- %
In this exercise, you have to create a \texttt{Vector} class. The goal is to
create vectors and be able to perform mathematical operations with them.
\begin{itemize}
  \item Column vectors are represented as list of lists of single float (\texttt{[[1.], [2.], [3.]]}),
  \item Row vectors are represented as a list of a list of several floats (\texttt{[[1., 2., 3.]]}).
\end{itemize}

\warn{A vector is either a single line of floats or a single column of floats. When more than a line/column is consider, it is a matrix, not a vector.}


The class should also has 2 attributes:
\begin{itemize}
  \item \texttt{values}: list of list of floats (for row vector) or list of lists of single float (for column vector),
  \item \texttt{shape}: tuple of 2 integers: \texttt{(1,n)} for a row vector of dimension $n$
  or \texttt{(n,1)} for a column vector of dimension $n$. 
\end{itemize}

\info{If you did not learn at school what is the dimension of a vector, don't worry.
But for now do not think too hard about what dimension means.
Just consider the dimension is the number of floats (elements/coordinates) of a vector, and shape gives the layout:
      if $(1, n)$ the vector is a row, if $(n,1)$ the vector is a column.}
      
Finally you have to implement 2 methods: 
\begin{itemize}
  \item \texttt{.dot()} produce a dot product between two vectors of same \textbf{shape},
  \item \texttt{.T()} returns the transpose vector (i.e. a column vector into a row vector, or a row vector into a column vector).
\end{itemize}

You will also provide a testing file (\texttt{test.py}) to demonstrate your class
works as expected. In this testing file, demonstrate:
\begin{itemize}
  \item the addition and substraction are working for 2 vectors of the same shape,
  \item the mutliplication (\texttt{mul} and \texttt{rmul}) are working for a vector and a scalar,
  \item the division (\texttt{truediv}) is working with a vector and a scalar,
  \item the division (\texttt{rtruediv}) raises an Arithmetic Error (this test can be commented for the other tests and uncommented to show this one),
\end{itemize}

% ================================= %
\section*{Examples}
% --------------------------------- %
\begin{minted}[bgcolor=darcula-back,formatcom=\color{lightgrey},fontsize=\scriptsize]{python}
# Column vector of shape n * 1
v1 = Vector([[0.0], [1.0], [2.0], [3.0]])
v2 = v1 * 5
print(v2)
# Expected output:
# Vector([[0.0], [5.0], [10.0], [15.0]])

# Row vector of shape 1 * n
v1 = Vector([[0.0, 1.0, 2.0, 3.0]])
v2 = v1 * 5
print(v2)
# Expected output
# Vector([[0.0, 5.0, 10.0, 15.0]])

v2 = v1 / 2.0
print(v2)
# Expected output
# Vector([[0.0], [0.5], [1.0], [1.5]])

v1 / 0.0
# Expected ouput
# ZeroDivisionError: division by zero.

2.0 / v1
# Expected output:
# NotImplementedError: Division of a scalar by a Vector is not defined here.
\end{minted}


\begin{minted}[bgcolor=darcula-back,formatcom=\color{lightgrey},fontsize=\scriptsize]{python}
# Column vector of shape (n, 1)
print(Vector([[0.0], [1.0], [2.0], [3.0]]).shape)
# Expected output
# (4,1)

print(Vector([[0.0], [1.0], [2.0], [3.0]]).values)
# Expected output
# [[0.0], [1.0], [2.0], [3.0]]

# Row vector of shape (1, n)
print(Vector([[0.0, 1.0, 2.0, 3.0]]).shape)
# Expected output
# (1,4)

print(Vector([[0.0, 1.0, 2.0, 3.0]]).values)
# Expected output
# [[0.0, 1.0, 2.0, 3.0]]
\end{minted}


\begin{minted}[bgcolor=darcula-back,formatcom=\color{lightgrey},fontsize=\scriptsize]{python}
# Example 1:
v1 = Vector([[0.0], [1.0], [2.0], [3.0]])
print(v1.shape)
# Expected output:
(4,1)

print(v1.T())
# Expected output:
# Vector([[0.0, 1.0, 2.0, 3.0]])

print(v1.T().shape)
# Expected output:
# (1,4)

# Example 2:
v2 = Vector([[0.0, 1.0, 2.0, 3.0]])
print(v2.shape)
# Expected output:
# (1,4)

print(v2.T())
# Expected output:
# Vector([[0.0], [1.0], [2.0], [3.0]])

print(v2.T().shape)
# Expected output:
# (4,1)
\end{minted}

\begin{minted}[bgcolor=darcula-back,formatcom=\color{lightgrey},fontsize=\scriptsize]{python}
  # Example 1:
  v1 = Vector([[0.0], [1.0], [2.0], [3.0]])
  v2 = Vector([[2.0], [1.5], [2.25], [4.0]])
  print(v1.dot(v2))
  # Expected output:
  # 18.0

  v3 = Vector([[1.0, 3.0]])
  v4 = Vector([[2.0, 4.0]])
  print(v3.dot(v4))
  # Expected output:
  # 13.0

  v1
  # Expected output: to see what __repr__() should do
  # [[0.0, 1.0, 2.0, 3.0]]

  print(v1)
  # Expected output: to see what __str__() should do
  # [[0.0, 1.0, 2.0, 3.0]]
\end{minted}

You should be able to initialize the object with:
\begin{itemize}
  \item a list of a list of floats: \texttt{Vector([[0.0, 1.0, 2.0, 3.0]])},
  \item a list of lists of single float: \texttt{Vector([[0.0], [1.0], [2.0], [3.0]])},
  \item a size: \texttt{Vector(3)} -> the vector will have \texttt{values = [[0.0], [1.0], [2.0]]},
  \item a range: \texttt{Vector((10,16))} -> the vector will have \texttt{values = [[10.0], [11.0], [12.0], [13.0], [14.0], [15.0]]}.
    in \texttt{Vector((a,b))}, if \texttt{a > b}, you must display accurate error message.
\end{itemize}

\textit{By default, the vectors are generated as classical column vectors if initialized with a size or range.}

To perform arithmetic operations for Vector-Vector or scalar-Vector, you have to implement all the following built-in functions (called \texttt{magic/special methods}) for your \texttt{Vector} class:

\begin{minted}[bgcolor=darcula-back,formatcom=\color{lightgrey},fontsize=\scriptsize]{python}
  __add__
  __radd__
  # add & radd : only vectors of same shape.
  __sub__
  __rsub__
  # sub & rsub: only vectors of same shape.
  __truediv__
  # truediv : only with scalars (to perform division of Vector by a scalar).
  __rtruediv__
  # rtruediv : raises an NotImplementedError with the message "Division of a scalar by a Vector is not defined here."
  __mul__
  __rmul__
  # mul & rmul: only scalars (to perform multiplication of Vector by a scalar).
  __str__
  __repr__
  # must be identical, i.e we expect that print(vector) and vector within python interpretor behave the same, see corresponding example section.
\end{minted}

\info{
  So it might be a good idea to implement \texttt{values} and \texttt{shape} before built-in arithmetic functions.
  For the case not specify (e.g vector * vector) you should raise \texttt{NotImplementedError}.
}


% ================================= %
\section*{Mathematic notions}
% --------------------------------- %
The authorized vector operations are:  

\begin{itemize}
  \item Addition between two vectors of same dimension $m$
  \begin{equation*}
    x + y  =  
    \begin{bmatrix} x_1 \\ \vdots \\ x_m\end{bmatrix} + 
    \begin{bmatrix} y_1 \\ \vdots \\ y_m\end{bmatrix} \\ 
     =  \begin{bmatrix} x_1 + y_1 \\ \vdots \\ x_m + y_m \end{bmatrix}
  \end{equation*}
  \item Subtraction between two vectors of same dimension $m$
  \begin{equation*}
    x - y = 
    \begin{bmatrix} x_1 \\ \vdots \\ y_m\end{bmatrix} - 
    \begin{bmatrix} x_1 \\ \vdots \\ y_m\end{bmatrix} 
    = \begin{bmatrix} x_1 - y_1 \\ \vdots \\ x_m - y_m \end{bmatrix}
  \end{equation*}
  \item  Multiplication and division between one vector $m$ and one scalar.
  \begin{equation*}
    \alpha x = \alpha \begin{bmatrix} x_1 \\ \vdots \\ x_m\end{bmatrix}  = 
    \begin{bmatrix} \alpha x_1 \\ \vdots \\ \alpha x_m \end{bmatrix}
  \end{equation*}
  \item  Dot product between two vectors of same dimension $m$  
  \begin{equation*}
    x \cdot y = \begin{bmatrix} x_1 \\ \vdots \\ x_m\end{bmatrix} 
    \cdot 
    \begin{bmatrix} y_1 \\ \vdots \\ y_m\end{bmatrix} = 
    \sum_{i = 1}^{m} x_i \cdot y_i =  x_1 \cdot y_1 + \dots + x_m \cdot y_m 
  \end{equation*}
\end{itemize}

Do not forget to handle all types of error properly!


% ===========================(fin ex 02)         %
% ============================================== %

\newpage

% ============================================== %
% ===========================(start ex 03)       %
\chapter{Exercise 03}
\extitle{Generator!}
\turnindir{ex03}
\exnumber{03}
\exfiles{generator.py}
\exauthorize{random.randint, random.choice}
\exforbidden{random.shuffle, random.sample}
\makeheaderfilesforbidden


% ================================= %
\section*{Objective}
% --------------------------------- %
The goal of the exercise is to discover the concept of generator object in Python.

% ================================= %
\section*{Instructions}
% --------------------------------- %
Code a function called \texttt{generator} that takes a text as input (only printable characters), uses the string
parameter \texttt{sep} as a splitting parameter, and \texttt{yield}s the resulting substrings.

The function can take an optional argument.
The options are:
\begin{itemize}
  \item \texttt{shuffle}: shuffles the list of words,
  \item \texttt{unique}: returns a list where each word appears only once,
  \item \texttt{ordered}: alphabetically sorts the words.
\end{itemize}

\begin{minted}[bgcolor=darcula-back,formatcom=\color{lightgrey},fontsize=\scriptsize]{python}
  # function prototype
  def generator(text, sep=" ", option=None):
    '''Splits the text according to sep value and yield the substrings.
       option precise if a action is performed to the substrings before it is yielded.
    '''
\end{minted}

You can only call one option at a time.

% ================================= %
\section*{Examples}
% --------------------------------- %
\begin{minted}[bgcolor=darcula-back,formatcom=\color{lightgrey},fontsize=\scriptsize]{python}
  >> text = "Le Lorem Ipsum est simplement du faux texte."
  >> for word in generator(text, sep=" "):
  ...     print(word)
  ...
  Le
  Lorem
  Ipsum
  est
  simplement
  du
  faux
  texte.

  >> for word in generator(text, sep=" ", option="shuffle"):
  ...     print(word)
  ...
  simplement
  texte.
  est
  faux
  Le
  Lorem
  Ipsum
  du

  >> for word in generator(text, sep=" ", option="ordered"):
  ...     print(word)
  ...
  Ipsum
  Le
  Lorem
  du
  est
  faux
  simplement
  texte.
\end{minted}

\begin{minted}[bgcolor=darcula-back,formatcom=\color{lightgrey},fontsize=\scriptsize]{python}
>> text = "Lorem Ipsum Lorem Ipsum"
>> for word in generator(text, sep=" ", option="unique"):
...     print(word)
...
Lorem
Ipsum
\end{minted}

The function should return "ERROR" one time if the \texttt{text} argument is not a string, or if the \texttt{option} argument is not valid.

\begin{minted}[bgcolor=darcula-back,formatcom=\color{lightgrey},fontsize=\scriptsize]{python}
>> text = 1.0
>> for word in generator(text, sep="."):
...     print(word)
...
ERROR
\end{minted}


% ===========================(fin ex 03)         %
% ============================================== %

\newpage

% ============================================== %
% ===========================(start ex 04)       %
\chapter{Exercise 04}
\extitle{Working with lists}
\turnindir{ex04}
\exnumber{04}
\exfiles{eval.py}
\exauthorize{zip and enumerate}
\exforbidden{while}
\makeheaderfilesforbidden


% ================================= %
\section*{Objective}
% --------------------------------- %
The goal of the exercise is to discover 2 useful methods for lists, tuples,
dictionnaries (iterable class objects more generally) named \texttt{zip}
and \texttt{enumerate}.

% ================================= %
\section*{Instructions}
% --------------------------------- %
Code a class \texttt{Evaluator}, that has two static functions named \texttt{zip\_evaluate}
and \texttt{enumerate\_evaluate}.


The goal of these 2 functions is to compute the sum of the lengths of every
\texttt{words} of a given list weighted by a list of coefficinents \texttt{coefs} (yes, the 2 functions should do the same thing).

The lists \texttt{coefs} and \texttt{words} have to be the same length. If this is not the
case, the function should return -1.

You have to obtain the desired result using \texttt{zip} in the \texttt{zip\_evaluate} function,
and with \texttt{enumerate} in the \texttt{enumerate\_evaluate} function.

% ================================= %
\section*{Examples}
% --------------------------------- %
\begin{minted}[bgcolor=darcula-back,formatcom=\color{lightgrey},fontsize=\scriptsize]{python}
>> from eval import Evaluator
>> 
>> words = ["Le", "Lorem", "Ipsum", "est", "simple"]
>> coefs = [1.0, 2.0, 1.0, 4.0, 0.5]
>> Evaluator.zip_evaluate(coefs, words)
32.0
>> words = ["Le", "Lorem", "Ipsum", "n'", "est", "pas", "simple"]
>> coefs = [0.0, -1.0, 1.0, -12.0, 0.0, 42.42]
>> Evaluator.enumerate_evaluate(coefs, words)
-1
\end{minted}


% ===========================(fin ex 04)         %
% ============================================== %

\newpage

% ============================================== %
% ===========================(start ex 05)       %
\chapter{Exercise 05}
\extitle{Bank Account}
\turnindir{ex05}
\exnumber{05}
\exfiles{the\_bank.py}
\exforbidden{None}
\makeheaderfilesforbidden

% ================================= %
\section*{Objective}
% --------------------------------- %
The goals of this exercise is to discover new built-in functions and
deepen your class manipulation and to be aware of possibility
to modify instanced objects.\\
In this exercise you learn how to modify or add attributes to an object.

% ================================= %
\section*{Instructions}
% --------------------------------- %
It is all about security.
Have a look at the class named \texttt{Account} in the snippet of code below.

\begin{minted}[bgcolor=darcula-back,formatcom=\color{lightgrey},fontsize=\scriptsize]{python}
# in the_bank.py
class Account(object):

    ID_COUNT = 1

    def __init__(self, name, **kwargs):
        self.__dict__.update(kwargs)
        
        self.id = self.ID_COUNT
        Account.ID_COUNT += 1
        self.name = name
        if not hasattr(self, 'value'):
            self.value = 0

        if self.value < 0:
          raise AttributeError("Attribute value cannot be negative.")
        if not isinstance(self.name, str)
          raise AttributeError("Attribute name must be a str object.")
    
    def transfer(self, amount):
        self.value += amount
\end{minted}

Now, it is your turn to code a class named \texttt{Bank}!
Its purpose will be to handle the security part of each transfer attempt.

Security means checking if the \texttt{Account} is:
\begin{itemize}
  \item the right object,
  \item not corrupted,
  \item and stores enough money to complete the transfer.
\end{itemize}

How do we define if a bank account is corrupted? A corrupted bank account has:
\begin{itemize}
  \item an even number of attributes,
  \item an attribute starting with \texttt{b},
  \item no attribute starting with \texttt{zip} or \texttt{addr},
  \item no attribute \texttt{name}, \texttt{id} and \texttt{value},
  \item \texttt{name} not being a string,
  \item \texttt{id} not being an \texttt{int},
  \item \texttt{value} not being an \texttt{int} or a \texttt{float}.
\end{itemize}

For the rest of the attributes (\texttt{addr}, \texttt{zip}, etc ...
there is no specific check expected.
Meaning you are not expected to evaluate the validity of the account based on the type of the other attributes (the conditions listed above are sufficient).

Moreover, verification has to be performed when account objects are added to to Bank instance 
(\texttt{bank.add(Account(...))}).
The verification in \texttt{add} only check the type of the new\_account and if there 
is no account among the one already in Bank instance with the same name.

A transaction is invalid if \texttt{amount < 0} or if the amount is larger than
the balance of the account.
Prior to the transfer, the validity of the 2 accounts (\texttt{origin} and \texttt{dest}) are checked
(according to the list of criteria above).
A transfer between the same account (\texttt{bank.transfer('Wiliam John', 'William John')})
is valid but there is no fund movement.

\texttt{fix\_account} recovers a corrupted account if it parameter \texttt{name} correspond to the attribute
name of one of the account in \texttt{accounts} (attribute of Bank). If name is not a string or does not corresponded
to an account name, the method return \texttt{False}.

\begin{minted}[bgcolor=darcula-back,formatcom=\color{lightgrey},fontsize=\scriptsize]{python}
# in the_bank.py
class Bank(object):
    """The bank"""
    def __init__(self):
        self.accounts = []

    def add(self, new_account):
        """ Add new_account in the Bank
            @new_account:  Account() new account to append
            @return   True if success, False if an error occured
        """
        # test if new_account is an Account() instance and if 
        # it can be appended to the attribute accounts
        
        # ... Your code  ...

        self.accounts.append(new_account)

    def transfer(self, origin, dest, amount):
        """" Perform the fund transfer
            @origin:  str(name) of the first account
            @dest:    str(name) of the destination account
            @amount:  float(amount) amount to transfer
            @return   True if success, False if an error occured
        """
        # ... Your code  ...

    def fix_account(self, name):
    """ fix account associated to name if corrupted
            @name:   str(name) of the account
            @return  True if success, False if an error occured
        """
        # ... Your code  ...
\end{minted}

Check out the \texttt{dir} built-in function.

\warn{YOU WILL HAVE TO MODIFY THE INSTANCES' ATTRIBUTES IN ORDER TO FIX THEM.}

% ================================= %
\section*{Examples}
% --------------------------------- %
The script \texttt{banking\_test1.py} is a test which must print \texttt{Failed}.
The second script \texttt{banking\_test2.py} is a test which must print \texttt{Failed} and then \texttt{Success}.

\begin{minted}[bgcolor=darcula-back,formatcom=\color{lightgrey},fontsize=\scriptsize]{bash}
>> python banking_test1.py
Failed
# The transaction is not performed has the account of Smith Jane is corrupted (due to the attribute 'bref').

>> python banking_test2.py
Failed
Success
# the account are false due to the abscence of addr attribute, fix_account recover the account,
# thus they become valid.
\end{minted}


% ===========================(fin ex 05)         %
% ============================================== %

\newpage

% ================================= %
\section*{Contact}
% --------------------------------- %
You can contact 42AI association by email: contact@42ai.fr\\
You can join the association on \href{https://join.slack.com/t/42-ai/shared_invite/zt-ebccw5r7-YPkDM6xOiYRPjqJXkrKgcA}{42AI slack}
and/or posutale to \href{https://forms.gle/VAFuREWaLmaqZw2D8}{one of the association teams}.

% ================================= %
\section*{Acknowledgements}
% --------------------------------- %
The modules Python \& ML is the result of a collective work, we would like to thanks:
\begin{itemize}
  \item Maxime Choulika (cmaxime),
  \item Pierre Peigné (ppeigne, pierre@42ai.fr),
  \item Matthieu David (mdavid, matthieu@42ai.fr),
  \item Quentin Feuillade--Montixi (qfeuilla, quentin@42ai.fr)
\end{itemize}
who supervised the creation, the enhancement and this present transcription.

\begin{itemize}
    \item Louis Develle (ldevelle, louis@42ai.fr)
    \item Augustin Lopez (aulopez)
    \item Luc Lenotre (llenotre)
    \item Owen Roberts (oroberts)
    \item Thomas Flahault (thflahau)
    \item Amric Trudel (amric@42ai.fr)
    \item Baptiste Lefeuvre (blefeuvr@student.42.fr)
    \item Mathilde Boivin (mboivin@student.42.fr)
    \item Tristan Duquesne (tduquesn@student.42.fr)
\end{itemize}
for your investment for the creation and development of these modules.

\begin{itemize}
    \item Barthélémy Leveque (bleveque@student.42.fr)
    \item Remy Oster (roster@student.42.fr)
    \item Quentin Bragard (qbragard@student.42.fr)
    \item Marie Dufourq (madufour@student.42.fr)
    \item Adrien Vardon (advardon@student.42.fr)
\end{itemize}
who betatest the first version of the modules of Machine Learning.
\vfill
\doclicenseThis

\end{document}