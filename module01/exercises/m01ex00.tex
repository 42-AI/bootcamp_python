\chapter{Exercise 00}
\extitle{The Book}
\turnindir{ex00}
\exnumber{00}
\exfiles{book.py, recipe.py, test.py}
\exforbidden{None}
\makeheaderfilesforbidden

% ================================== %
\section*{Objective}
% ---------------------------------- %
The goal of this exercise is to get you familiar with the notions of
classes and the manipulation of the objects related to these classes.

% ================================== %
\section*{Instructions}
% ---------------------------------- %
You will have to make a class \texttt{Book} and a class \texttt{Recipe}.
The classes \texttt{Book} and \texttt{Recipe} will be written in 
\texttt{book.py} and \texttt{recipe.py} respectively.\\
\newline
Let's describe the \texttt{Recipe} class. It has some attributes:
\begin{itemize}
  \item \texttt{name} (str): name of the recipe,
  \item \texttt{cooking\_lvl} (int): range from $1$ to $5$,
  \item \texttt{cooking\_time} (int): in minutes (no negative numbers),
  \item \texttt{ingredients} (list): list of all ingredients each represented by a string,
  \item \texttt{description} (str): description of the recipe,
  \item \texttt{recipe\_type} (str): can be "starter", "lunch" or "dessert".
\end{itemize}
You have to \textbf{initialize} the object \texttt{Recipe} and check all of its values. Only the description can be empty.
In case of input errors, you should print what they are and exit properly.\\
\newline
You will have to implement the built-in method \texttt{\_\_str\_\_}.
It's the method called when the following code is executed:\\
\newline
\begin{minted}[bgcolor=darcula-back,formatcom=\color{lightgrey},fontsize=\scriptsize]{python}
tourte = Recipe(...)
to_print = str(tourte)
print(to_print)
\end{minted}
\newline
\newline
It is implemented this way:\\
\newline
\begin{minted}[bgcolor=darcula-back,formatcom=\color{lightgrey},fontsize=\scriptsize]{python}
def __str__(self):
    """Returns the string to print with the recipe's info"""
    txt = ""
    """Your code here"""
    return txt
\end{minted}
\newline
The \texttt{Book} class also has some attributes:
\begin{itemize}
  \item \texttt{name} (str): name of the book,
  \item \texttt{last\_update} \href{https://docs.python.org/3/library/datetime.html}{(datetime)}: the date of the last update,
  \item \texttt{creation\_date} \href{https://docs.python.org/3/library/datetime.html}{(datetime)}: the creation date of the book,
  \item \texttt{recipes\_list} (dict): a dictionnary with 3 keys: "starter", "lunch", "dessert".
\end{itemize}
You will have to implement some methods in the \texttt{Book} class:\\
\newline
\begin{minted}[bgcolor=darcula-back	,formatcom=\color{lightgrey},fontsize=\scriptsize]{python}
def get_recipe_by_name(self, name):
    """Prints a recipe with the name \texttt{name} and returns the instance"""
    #... Your code here ...

def get_recipes_by_types(self, recipe_type):
    """Gets all recipes names for a given recipe_type """
    #... Your code here ...

def add_recipe(self, recipe):
    """Adds a recipe to the book and updates last_update"""
    #... Your code here ...
\end{minted}
\newline
You have to handle the error if the argument passed in \texttt{add\_recipe} is not a \texttt{Recipe}.\\
\newline
Finally, you will provide a \texttt{test.py} file to test your classes and prove that they are working properly.
You can import all the classes into your \texttt{test.py} file by adding these lines at the top of the \texttt{test.py} file:\\
\newline
\begin{minted}[bgcolor=darcula-back	,formatcom=\color{lightgrey},fontsize=\scriptsize]{python}
from book import Book
from recipe import Recipe

# ... Your tests ...
\end{minted}