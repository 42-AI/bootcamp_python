\chapter{Exercise 02}
\extitle{The Vector}
\turnindir{ex02}
\exnumber{02}
\exfiles{vector.py, test.py}
\exforbidden{Numpy library}
\makeheaderfilesforbidden

% ================================= %
\section*{Objective}
% --------------------------------- %
The goal of this exercise is to get you used to working with built-in methods, more
particularly with those allowing to perform operations.
You are expected to code built-in methods for vector-vector and
vector-scalar operations as rigorously as possible.

% ================================= %
\section*{Instructions}
% --------------------------------- %
In this exercise, you have to create a \texttt{Vector} class. The goal is to
create vectors and be able to perform mathematical operations with them.
\begin{itemize}
  \item Column vectors are represented as list of lists of single float (\texttt{[[1.], [2.], [3.]]}),
  \item Row vectors are represented as a list of a list of several floats (\texttt{[[1., 2., 3.]]}).
\end{itemize}

\warn{A vector is either a single line of floats or a single column of floats. When more than a line/column is considered, it is a matrix, not a vector.}


The class should also have 2 attributes:
\begin{itemize}
  \item \texttt{values}: list of list of floats (for row vector) or list of lists of single float (for column vector),
  \item \texttt{shape}: tuple of 2 integers: \texttt{(1,n)} for a row vector of dimension $n$
  or \texttt{(n,1)} for a column vector of dimension $n$. 
\end{itemize}

\info{If you did not learn at school what is the dimension of a vector, don't worry.
But for now do not think too hard about what dimension means.
Just consider the dimension is the number of floats (elements/coordinates) of a vector, and shape gives the layout:
      if $(1, n)$ the vector is a row, if $(n,1)$ the vector is a column.}
      
Finally you have to implement 2 methods: 
\begin{itemize}
  \item \texttt{.dot()} produces a dot product between two vectors of same \textbf{shape},
  \item \texttt{.T()} returns the transpose vector (i.e. a column vector into a row vector, or a row vector into a column vector).
\end{itemize}

You will also provide a test file (\texttt{test.py}) to demonstrate your class
works as expected. In this test file, you will demonstrate:
\begin{itemize}
  \item the addition and substraction are working for 2 vectors of the same shape,
  \item the multiplication (\texttt{mul} and \texttt{rmul}) are working for a vector and a scalar,
  \item the division (\texttt{truediv}) is working with a vector and a scalar,
  \item the division (\texttt{rtruediv}) raises an Arithmetic Error (this test can be commented for the other tests and uncommented to show this one),
\end{itemize}

\section*{Examples}
% --------------------------------- %
\begin{minted}[bgcolor=darcula-back,formatcom=\color{lightgrey},fontsize=\scriptsize]{python}
# Column vector of shape n * 1
v1 = Vector([[0.0], [1.0], [2.0], [3.0]])
v2 = v1 * 5
print(v2)
# Expected output:
# Vector([[0.0], [5.0], [10.0], [15.0]])

# Row vector of shape 1 * n
v1 = Vector([[0.0, 1.0, 2.0, 3.0]])
v2 = v1 * 5
print(v2)
# Expected output
# Vector([[0.0, 5.0, 10.0, 15.0]])

v2 = v1 / 2.0
print(v2)
# Expected output
# Vector([[0.0, 0.5, 1.0, 1.5]])

v1 / 0.0
# Expected ouput
# ZeroDivisionError: division by zero.

2.0 / v1
# Expected output:
# NotImplementedError: Division of a scalar by a Vector is not defined here.
\end{minted}

\begin{minted}[bgcolor=darcula-back,formatcom=\color{lightgrey},fontsize=\scriptsize]{python}
# Column vector of shape (n, 1)
print(Vector([[0.0], [1.0], [2.0], [3.0]]).shape)
# Expected output
# (4,1)

print(Vector([[0.0], [1.0], [2.0], [3.0]]).values)
# Expected output
# [[0.0], [1.0], [2.0], [3.0]]

# Row vector of shape (1, n)
print(Vector([[0.0, 1.0, 2.0, 3.0]]).shape)
# Expected output
# (1,4)

print(Vector([[0.0, 1.0, 2.0, 3.0]]).values)
# Expected output
# [[0.0, 1.0, 2.0, 3.0]]
\end{minted}


\begin{minted}[bgcolor=darcula-back,formatcom=\color{lightgrey},fontsize=\scriptsize]{python}
# Example 1:
v1 = Vector([[0.0], [1.0], [2.0], [3.0]])
print(v1.shape)
# Expected output:
(4,1)

print(v1.T())
# Expected output:
# Vector([[0.0, 1.0, 2.0, 3.0]])

print(v1.T().shape)
# Expected output:
# (1,4)

# Example 2:
v2 = Vector([[0.0, 1.0, 2.0, 3.0]])
print(v2.shape)
# Expected output:
# (1,4)

print(v2.T())
# Expected output:
# Vector([[0.0], [1.0], [2.0], [3.0]])

print(v2.T().shape)
# Expected output:
# (4,1)
\end{minted}

\begin{minted}[bgcolor=darcula-back,formatcom=\color{lightgrey},fontsize=\scriptsize]{python}
  # Example 1:
  v1 = Vector([[0.0], [1.0], [2.0], [3.0]])
  v2 = Vector([[2.0], [1.5], [2.25], [4.0]])
  print(v1.dot(v2))
  # Expected output:
  # 18.0

  v3 = Vector([[1.0, 3.0]])
  v4 = Vector([[2.0, 4.0]])
  print(v3.dot(v4))
  # Expected output:
  # 14.0

  v1
  # Expected output: to see what __repr__() should do
  # [[0.0, 1.0, 2.0, 3.0]]

  print(v1)
  # Expected output: to see what __str__() should do
  # [[0.0, 1.0, 2.0, 3.0]]
\end{minted}

You should be able to initialize the object with:
\begin{itemize}
  \item a list of a list of floats: \texttt{Vector([[0.0, 1.0, 2.0, 3.0]])},
  \item a list of lists of single float: \texttt{Vector([[0.0], [1.0], [2.0], [3.0]])},
  \item a size: \texttt{Vector(3)} -> the vector will have \texttt{values = [[0.0], [1.0], [2.0]]},
  \item a range: \texttt{Vector((10,16))} -> the vector will have \texttt{values = [[10.0], [11.0], [12.0], [13.0], [14.0], [15.0]]}.
    in \texttt{Vector((a,b))}, if \texttt{a > b}, you must display accurate error message.
\end{itemize}

\textit{By default, the vectors are generated as classical column vectors if initialized with a size or range.}

To perform arithmetic operations for Vector-Vector or Scalar-Vector, you have to implement all the following built-in functions (called \texttt{magic/special methods}) for your \texttt{Vector} class:

\begin{minted}[bgcolor=darcula-back,formatcom=\color{lightgrey},fontsize=\scriptsize]{python}
  __add__
  __radd__
  # add & radd : only vectors of the same shape.
  __sub__
  __rsub__
  # sub & rsub: only vectors of the same shape.
  __truediv__
  # truediv : only with scalars (to perform division of a Vector by a scalar).
  __rtruediv__
  # rtruediv : raises an NotImplementedError with the message "Division of a scalar by a Vector is not defined here."
  __mul__
  __rmul__
  # mul & rmul: only scalars (to perform multiplication of a Vector by a scalar).
  __str__
  __repr__
  # must be identical, i.e we expect that print(vector) and vector within python interpretor to behave the same, see corresponding example section.
\end{minted}

\info{
  So it might be a good idea to implement \texttt{values} and \texttt{shape} before built-in arithmetic functions.
  For the case not specified (e.g vector * vector) you should raise \texttt{NotImplementedError}.
}


% ================================= %
\section*{Mathematic notions}
% --------------------------------- %
The authorized vector operations are:  

\begin{itemize}
  \item Addition between two vectors of same dimension $m$
  \begin{equation*}
    x + y  =  
    \begin{bmatrix} x_1 \\ \vdots \\ x_m\end{bmatrix} + 
    \begin{bmatrix} y_1 \\ \vdots \\ y_m\end{bmatrix} \\ 
     =  \begin{bmatrix} x_1 + y_1 \\ \vdots \\ x_m + y_m \end{bmatrix}
  \end{equation*}
  \item Substraction between two vectors of same dimension $m$
  \begin{equation*}
    x - y = 
    \begin{bmatrix} x_1 \\ \vdots \\ y_m\end{bmatrix} - 
    \begin{bmatrix} x_1 \\ \vdots \\ y_m\end{bmatrix} 
    = \begin{bmatrix} x_1 - y_1 \\ \vdots \\ x_m - y_m \end{bmatrix}
  \end{equation*}
  \item  Multiplication and division between one vector $m$ and one scalar.
  \begin{equation*}
    \alpha x = \alpha \begin{bmatrix} x_1 \\ \vdots \\ x_m\end{bmatrix}  = 
    \begin{bmatrix} \alpha x_1 \\ \vdots \\ \alpha x_m \end{bmatrix}
  \end{equation*}
  \item  Dot product between two vectors of same dimension $m$  
  \begin{equation*}
    x \cdot y = \begin{bmatrix} x_1 \\ \vdots \\ x_m\end{bmatrix} 
    \cdot 
    \begin{bmatrix} y_1 \\ \vdots \\ y_m\end{bmatrix} = 
    \sum_{i = 1}^{m} x_i \cdot y_i =  x_1 \cdot y_1 + \dots + x_m \cdot y_m 
  \end{equation*}
\end{itemize}

Do not forget to handle all types of error properly!


% ===========================(fin ex 02)         %
% ============================================== %

