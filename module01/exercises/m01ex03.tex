\chapter{Exercise 03}
\extitle{Generator!}
\turnindir{ex03}
\exnumber{03}
\exfiles{generator.py}
\exauthorize{random.randint, random.choice}
\exforbidden{random.shuffle, random.sample}
\makeheaderfilesforbidden


% ================================= %
\section*{Objective}
% --------------------------------- %
The goal of this exercise is to discover the concept of generator object in Python.

% ================================= %
\section*{Instructions}
% --------------------------------- %
Code a function called \texttt{generator} that takes a text as input (only printable characters), uses the string
parameter \texttt{sep} as a splitting parameter, and \texttt{yield}s the resulting substrings.

The function can take an optional argument.
The options are:
\begin{itemize}
  \item \texttt{shuffle}: shuffles the list of words,
  \item \texttt{unique}: returns a list where each word appears only once,
  \item \texttt{ordered}: sorts the words alphabetically.
\end{itemize}

\begin{minted}[bgcolor=darcula-back,formatcom=\color{lightgrey},fontsize=\scriptsize]{python}
  # function prototype
  def generator(text, sep=" ", option=None):
    '''
    Splits the text according to sep value and yields the substrings.
    '''
\end{minted}

You can only call one option at a time.

% ================================= %
\section*{Examples}
% --------------------------------- %
\begin{minted}[bgcolor=darcula-back,formatcom=\color{lightgrey},fontsize=\scriptsize]{python}
  >> text = "Le Lorem Ipsum est simplement du faux texte."
  >> for word in generator(text, sep=" "):
  ...     print(word)
  ...
  Le
  Lorem
  Ipsum
  est
  simplement
  du
  faux
  texte.

  >> for word in generator(text, sep=" ", option="shuffle"):
  ...     print(word)
  ...
  simplement
  texte.
  est
  faux
  Le
  Lorem
  Ipsum
  du

  >> for word in generator(text, sep=" ", option="ordered"):
  ...     print(word)
  ...
  Ipsum
  Le
  Lorem
  du
  est
  faux
  simplement
  texte.
\end{minted}

\begin{minted}[bgcolor=darcula-back,formatcom=\color{lightgrey},fontsize=\scriptsize]{python}
>> text = "Lorem Ipsum Lorem Ipsum"
>> for word in generator(text, sep=" ", option="unique"):
...     print(word)
...
Lorem
Ipsum
\end{minted}

The function should return "ERROR" one time if the \texttt{text} argument is not a string, or if the \texttt{option} argument is not valid.

\begin{minted}[bgcolor=darcula-back,formatcom=\color{lightgrey},fontsize=\scriptsize]{python}
>> text = 1.0
>> for word in generator(text, sep="."):
...     print(word)
...
ERROR
\end{minted}