% vim: set ts=4 sw=4 tw=80 noexpandtab:

\documentclass{42-en}

%******************************************************************************%
%                                                                              %
%                                   Prologue                                   %
%                                                                              %
%******************************************************************************%
\usepackage[
    type={CC},
    modifier={by-nc-sa},
    version={4.0},
]{doclicense}
\usemintedstyle{monokai}


%****************************************************************%
%                  Re/definition of commands                     %
%****************************************************************%

\newcommand{\ailogo}[1]{\def \@ailogo {#1}}\ailogo{assets/42ai_logo.pdf}

%%  Redefine \maketitle
\makeatletter
\def \maketitle {
  \begin{titlepage}
    \begin{center}
	%\begin{figure}[t]
	  %\includegraphics[height=8cm]{\@ailogo}
	  \includegraphics[height=8cm]{assets/42ai_logo.pdf}
	%\end{figure}
      \vskip 5em
      {\huge \@title}
      \vskip 2em
      {\LARGE \@subtitle}
      \vskip 4em
    \end{center}
    %\begin{center}
	  %\@author
    %\end{center}
	%\vskip 5em
  \vfill
  \begin{center}
    \emph{\summarytitle : \@summary}
  \end{center}
  \vspace{2cm}
  %\vskip 5em
  %\doclicenseThis
  \end{titlepage}
}
\makeatother

\makeatletter
\def \makeheaderfilesforbidden
{
  \noindent
  \begin{tabularx}{\textwidth}{|X X  X X|}
    \hline
  \multicolumn{1}{|>{\raggedright}m{1cm}|}
  {\vskip 2mm \includegraphics[height=1cm]{assets/42ai_logo.pdf}} &
  \multicolumn{2}{>{\centering}m{12cm}}{\small Exercise : \@exnumber } &
  \multicolumn{1}{ >{\raggedleft}p{1.5cm}|}
%%              {\scriptsize points : \@exscore} \\ \hline
              {} \\ \hline

  \multicolumn{4}{|>{\centering}m{15cm}|}
              {\small \@extitle} \\ \hline

  \multicolumn{4}{|>{\raggedright}m{15cm}|}
              {\small Turn-in directory : \ttfamily
                $ex\@exnumber/$ }
              \\ \hline
  \multicolumn{4}{|>{\raggedright}m{15cm}|}
              {\small Files to turn in : \ttfamily \@exfiles }
              \\ \hline

  \multicolumn{4}{|>{\raggedright}m{15cm}|}
              {\small Forbidden functions : \ttfamily \@exforbidden }
              \\ \hline

%%  \multicolumn{4}{|>{\raggedright}m{15cm}|}
%%              {\small Remarks : \ttfamily \@exnotes }
%%              \\ \hline
\end{tabularx}
%% \exnotes
\exrules
\exmake
\exauthorize{None}
\exforbidden{None}
\extitle{}
\exnumber{}
}
\makeatother


\begin{document}

% =============================================================================%
%                     =====================================                    %

\title{Python \& ML - Module 00}
\subtitle{Basic stuff - Eleven Commandments}
\author{
	Maxime Choulika (cmaxime), Pierre Peigné (ppeigne), Matthieu David (mdavid)
}

\summary
{
	The goal of the module is to get started with the Python language.
}

\maketitle
%******************************************************************************%
%                                                                              %
%                        Common Instructions                                   %
%                          for Python Projects                                 %
%                                                                              %
%******************************************************************************%

\chapter{Common Instructions}
\begin{itemize}
  \item The version of Python recommended to use is 3.7, you can
  check the version of Python with the following command: \texttt{python -V}
  
  \item The norm: during this piscine, it is recommended to follow the
  \href{https://www.python.org/dev/peps/pep-0008/}{PEP 8 standards}, though it is not mandatory.
  You can install \href{https://pypi.org/project/pycodestyle}{pycodestyle} which
  is a tool to check your Python code.
  \item The function \texttt{eval} is never allowed.
  \item The exercises are ordered from the easiest to the hardest.
  \item Your exercises are going to be evaluated by someone else,
  so make sure that your variable names and function names are appropriate and civil.
  \item Your manual is the internet.
  
  \item You can also ask questions in the \texttt{\#bootcamps} channel in the \href{https://42-ai.slack.com}{42AI}
  or \href{42born2code.slack.com}{42born2code}.
  
  \item If you find any issue or mistake in the subject please create an issue on \href{https://github.com/42-AI/bootcamp_python/issues}{42AI repository on Github}.
  
  \item We encourage you to create test programs for your
  project even though this work \textbf{won't have to be
  submitted and won't be graded}. It will give you a chance
  to easily test your work and your peers’ work. You will find
  those tests especially useful during your defence. Indeed,
  during defence, you are free to use your tests and/or the
  tests of the peer you are evaluating.
  
  \item Submit your work to your assigned git repository. Only the work in the
  git repository will be graded. If Deepthought is assigned to grade your
  work, it will be run after your peer-evaluations.
  If an error happens in any section of your work during Deepthought's grading,
  the evaluation will stop.
\end{itemize}
\newpage
\tableofcontents
\startexercices

%                     =====================================                    %
% =============================================================================%


%******************************************************************************%
%                                                                              %
%                                   Exercises                                  %
%                                                                              %
%******************************************************************************%

% ============================================== %
\chapter{Exercise 00}
\extitle{\$PATH}
\turnindir{ex00}
\exnumber{00}
\exfiles{answers.txt, requirements.txt}
\exforbidden{None}
\makeheaderfilesforbidden

\emph{The first thing you need is a proper working environment.}

Most modern Unix-based system have a \texttt{python} interpreter installed by default,
but its version might be lower/higher than the one used for these modules.
It is also possible that the default \texttt{python} command uses a version 2.X (for legacy reasons).
This is obviously very confusing for new developper.

\begin{42console}
$> python -V
$> python3 -V
\end{42console}

To deal with those version issues we will use \texttt{conda}. This program allow you to manage your Python packages and different working environments.

\emph{The requirement is to use a Python 3.7.X version. You are free to use a different program/utilities to achieve this goal. At your own risk.}

% ================================== %
\section*{Conda manual install}
% ---------------------------------- %

If you want a fully automated install go to \textbf{Conda automated install} section.
The automated part will allow you to reinstall everything more easily in case you use another computer. Below is a step by step installation.

We recommend the following path for your \texttt{conda} folder.

\begin{42console}
$> MYPATH="/goinfre/$USER/miniconda3"
\end{42console}

% ================================== %
\subsection*{1. Download \& Install conda}
% ---------------------------------- %
\begin{42console}
# For MAC
$> curl -LO "https://repo.anaconda.com/miniconda/Miniconda3-latest-MacOSX-x86_64.sh"
$> sh Miniconda3-latest-MacOSX-x86_64.sh -b -p $MYPATH

# For Linux
$> curl -LO "https://repo.anaconda.com/miniconda/Miniconda3-latest-Linux-x86_64.sh"
$> sh Miniconda3-latest-Linux-x86_64.sh -b -p $MYPATH
\end{42console}

% ================================== %
\subsection*{2. Initial configuration of conda}
% ---------------------------------- %
\begin{42console}
# For zsh
$> $MYPATH/bin/conda init zsh
$> $MYPATH/bin/conda config --set auto_activate_base false
$> source ~/.zshrc

# For bash
$> $MYPATH/bin/conda init bash
$> $MYPATH/bin/conda config --set auto_activate_base false
$> source ~/.bash_profile
\end{42console}

% ================================== %
\subsection*{3. Create an environment for 42AI !}
% ---------------------------------- %
\begin{42console}
$> conda create --name 42AI-$USER python=3.7 jupyter pandas pycodestyle
\end{42console}

% ================================== %
\subsection*{4. Check your 42AI Python environment}
% ---------------------------------- %
\begin{42console}
$> conda info --envs
$> conda activate 42AI-$USER
$> which python
$> python -V
$> python -c "print('Hello World!')"
\end{42console}

% ================================== %
\subsection*{Help !}
% ---------------------------------- %
\begin{itemize}
	\item \textbf{I have lost my miniconda3 folder !} Repeat step 1 and 3.
	\item \textbf{I have lost my home directory !} Repeat step 2.
\end{itemize}


\newpage
% ================================== %
\section*{Conda automated install}
% ---------------------------------- %

% ================================== %
\subsection*{Copy and launch the following shell script}
% ---------------------------------- %

\begin{42console}
#!/bin/bash

function which_dl {
	# If operating system name contains Darwnin: MacOS. Else Linux
	if uname -s | grep -iqF Darwin; then
		echo "Miniconda3-latest-MacOSX-x86_64.sh"
	else
		echo "Miniconda3-latest-Linux-x86_64.sh"
	fi
}

function which_shell {
	# if $SHELL contains zsh, zsh. Else Bash
	if echo $SHELL | grep -iqF zsh; then
		echo "zsh"
	else
		echo "bash"
	fi
}

function when_conda_exist {
	# check and install 42AI environement
	printf "Checking 42AI-$USER environment: "
	if conda info --envs | grep -iqF 42AI-$USER; then
		printf "\e[33mDONE\e[0m\n"
	else
		printf "\e[31mKO\e[0m\n"
		printf "\e[33mCreating 42AI environnment:\e[0m\n"
		conda update -n base -c defaults conda -y
		conda create --name 42AI-$USER python=3.7 jupyter numpy pandas pycodestyle -y
	fi
}

function set_conda {
	MINICONDA_PATH="/goinfre/$USER/miniconda3"
	CONDA=$MINICONDA_PATH"/bin/conda"
	PYTHON_PATH=$(which python)
	REQUIREMENTS="jupyter numpy pandas pycodestyle"
	SCRIPT=$(which_dl)
	MY_SHELL=$(which_shell)
	DL_LINK="https://repo.anaconda.com/miniconda/"$SCRIPT
	DL_LOCATION="/tmp/"

	printf "Checking conda: "
	TEST=$(conda -h 2>/dev/null)
	if [ $? == 0 ] ; then
		printf "\e[32mOK\e[0m\n"
		when_conda_exist
		return
	fi
	printf "\e[31mKO\e[0m\n"
	if [ ! -f $DL_LOCATION$SCRIPT ]; then
		printf "\e[33mDonwloading installer:\e[0m\n"
		cd $DL_LOCATION
		curl -LO $DL_LINK
		cd -
	fi
	printf "\e[33mInstalling conda:\e[0m\n"
	sh $DL_LOCATION$SCRIPT -b -p $MINICONDA_PATH
	printf "\e[33mConda initial setup:\e[0m\n"
	$CONDA init $MY_SHELL
	$CONDA config --set auto_activate_base false

	printf "\e[33mCreating 42AI-$USER environnment:\e[0m\n"
	$CONDA update -n base -c defaults conda -y
	$CONDA create --name 42AI-$USER python=3.7 jupyter numpy pandas pycodestyle -y
	printf "\e[33mLaunch the following command or restart your shell:\e[0m\n"
	if [ $MY_SHELL == "zsh" ]; then
		printf "\tsource ~/.zshrc\n"
	else
		printf "\tsource ~/.bash_profile\n"
	fi
}

set_conda
\end{42console}

% ================================== %
\subsection*{Reset your shell configuration}
% ---------------------------------- %
\begin{42console}
$> source ~/.zshrc
\end{42console}
\emph{Or close and reopen your shell.}

% ================================== %
\subsection*{Check your 42AI Python environment}
% ---------------------------------- %
\begin{42console}
$> conda info --envs
$> conda activate 42AI-$USER
$> which python
$> python -V
$> python -c "print('Hello World!')"
\end{42console}

\newpage
% ================================= %
\section*{Getting started}
% --------------------------------- %


Your \texttt{Python} environment is ready. Complete the following questionnaire using \texttt{python}, \texttt{pip} or \texttt{conda}. Save your answers in a file \texttt{answers.txt} (one answer per line), and check them with your peers.

Find the commands to:
\begin{itemize}\itemsep7pt
	\item Output a list of installed packages and their versions.
	\item Show the package metadata of \texttt{numpy}.
	\item Uninstall the package \texttt{numpy}.
	\item Install the package \texttt{numpy}.
	\item Freeze the packages and their current versions in a \texttt{requirements.txt} file you have to turn-in.
\end{itemize}

% ===========================(fin ex 00)         %
% ============================================== %
\newpage

% ============================================== %
% ===========================(start ex 01)       %
\chapter{Exercise 01}
\extitle{Rev Alpha}
\turnindir{ex01}
\exnumber{01}
\exfiles{exec.py}
\exforbidden{None}
\makeheaderfilesforbidden


% ================================= %
% \section*{}
% --------------------------------- %
Make a program that takes a string as argument, reverses it, swaps its letters case and print the result.

If more than one argument are provided, merge them into a single string with each argument separated by a space character.

If no argument are provided, do nothing or print an usage.

% ================================= %
\section*{Examples}
% --------------------------------- %
\begin{42console}
$> python3 exec.py 'Hello World!' | cat -e
!DLROw OLLEh$
$>
$> python3 exec.py 'Hello' 'my Friend' | cat -e
DNEIRf YM OLLEh$
$>
$> python3 exec.py
$>
\end{42console}

% ===========================(fin ex 01)         %
% ============================================== %

\newpage

% ============================================== %
% ===========================(start ex 02)       %
\chapter{Exercise 02}
\extitle{The Odd, the Even and the Zero}
\turnindir{ex02}
\exnumber{02}
\exfiles{whois.py}
\exforbidden{None}
\makeheaderfilesforbidden


% ================================= %
% \section*{}
% --------------------------------- %
Make a program that takes a number as argument, checks whether it is odd, even or zero, and print the result.

If more than one argument are provided or if the argument is not an integer, print an error message.

If no argument are provided, do nothing or print an usage.

% ================================= %
\section*{Examples}
% --------------------------------- %
\begin{42console}
$> python3 whois.py 12
I'm Even.
$>
$> python3 whois.py 3
I'm Odd.
$>
$> python3 whois.py
$>
$> python3 whois.py 0
I'm Zero.
$>
$> python3 whois.py Hello
AssertionError: argument is not an integer
$>
$> python3 whois.py 12 3
AssertionError: more than one argument are provided
$>
\end{42console}

\hint{
	you can handle errors as you want but take a look at \texttt{assert}
}

% ===========================(fin ex 02)         %
% ============================================== %

\newpage

% ============================================== %
% ===========================(start ex 03)       %
\chapter{Exercise 03}
\extitle{Functional file}
\turnindir{ex03}
\exnumber{03}
\exfiles{count.py}
\exforbidden{None}
\makeheaderfilesforbidden


% ================================= %
% \section*{}
% --------------------------------- %
Create a function called \texttt{text\_analyzer} that displays the sums of
upper-case characters, lower-case characters, punctuation characters and
spaces in a given text.


\texttt{text\_analyzer} will take only one parameter: the text to analyze.
You have to handle the case where the text is empty (maybe by setting a default value).
If there is no text passed to the function, the user is prompted to give one.


Test it in the Python console.

% ================================= %
\section*{Examples}
% --------------------------------- %
\begin{42console}
	$> python
	>>> from count import text_analyzer
	>>> text_analyzer("Python 2.0, released 2000, introduced
	features like List comprehensions and a garbage collection
	system capable of collecting reference cycles.")
	The text contains 143 characters:
	- 2 upper letters
	- 113 lower letters
	- 4 punctuation marks
	- 18 spaces

	>>> text_analyzer("Python is an interpreted, high-level,
	general-purpose programming language. Created by Guido van
	Rossum and first released in 1991, Python's design philosophy
	emphasizes code readability with its notable use of significant
	whitespace.")
	The text contains 234 characters:
	- 5 upper letters
	- 187 lower letters
	- 8 punctuation marks
	- 30 spaces

	>>> text_analyzer()
	What is the text to analyse?
	>> Python is an interpreted, high-level, general-purpose
	programming language. Created by Guido van Rossum and first
	released in 1991, Python's design philosophy emphasizes code
	readability with its notable use of significant whitespace.
	The text contains 234 characters:
	- 5 upper letters
	- 187 lower letters
	- 8 punctuation marks
	- 30 spaces
\end{42console}

Handle the case when more than one parameter is given to \texttt{text\_analyzer}:

\begin{42console}
	>>> from count import text_analyzer
	>>> text_analyzer("Python", "2.0")
	ERROR
\end{42console}

You're free to write your docstring and format it the way you want.

\begin{42console}
	>>> print(text_analyzer.__doc__)
    This function counts the number of upper characters, lower characters,
    punctuation and spaces in a given text.
\end{42console}


% ===========================(fin ex 03)         %
% ============================================== %

\newpage

% ============================================== %
% ===========================(start ex 04)       %
\chapter{Exercise 04}
\extitle{Elementary}
\turnindir{ex04}
\exnumber{04}
\exfiles{operations.py}
\exforbidden{None}
\makeheaderfilesforbidden


% ================================= %
% \section*{}
% --------------------------------- %
You have to make a program that prints the results of the four elementary
mathematical operations of arithmetic (addition, subtraction, multiplication,
division) and the modulo operation.
This should be accomplished by writing a function that takes 2 numbers as
parameters and returns 5 values, as formatted in the console output below.

% ================================= %
\section*{Examples}
% --------------------------------- %
\begin{42console}
	$> python operations.py 10 3
	Sum:         13
	Difference:  7
	Product:     30
	Quotient:    3.3333333333333335
	Remainder:   1
	$>
	$> python operations.py 42 10
	Sum:         52
	Difference:  32
	Product:     420
	Quotient:    4.2
	Remainder:   2
	$>
	$> python operations.py 1 0
	Sum:         1
	Difference:  1
	Product:     0
	Quotient:    ERROR (div by zero)
	Remainder:   ERROR (modulo by zero)
	$>
	$> python operations.py
	Usage: python operations.py <number1> <number2>
	Example:
		python operations.py 10 3
	$>
	$> python operations.py 12 10 5
	InputError: too many arguments

	Usage: python operations.py <number1> <number2>
	Example:
		python operations.py 10 3
	$>
	$> python operations.py "one" "two"
	InputError: only numbers

	Usage: python operations.py <number1> <number2>
	Example:
		python operations.py 10 3
	$>
	$> python operations.py "512" "63.1"
	InputError: only numbers

	Usage: python operations.py <number1> <number2>
	Example:
		python operations.py 10 3
\end{42console}

% ===========================(fin ex 04)         %
% ============================================== %

\newpage

% ============================================== %
% ===========================(start ex 05)       %
\chapter{Exercise 05}
\extitle{The right format}
\turnindir{ex05}
\exnumber{05}
\exfiles{kata00.py, kata01.py, kata02.py, kata03.py, kata04.py}
\exforbidden{None}
\makeheaderfilesforbidden

Let's get familiar with the useful concept of \textbf{string formatting}
 through a kata series.

% ================================= %
\section*{kata00}
% --------------------------------- %
% PYTHON ENV SERAIT MIEUX
\begin{42console}
	t = (19,42,21)
\end{42console}

Including the tuple above in your file, write a program that dynamically
builds up a formatted string like the following:

\begin{42console}
	$> python kata00.py
	The 3 numbers are: 19, 42, 21
\end{42console}

% ================================= %
\section*{kata01}
% --------------------------------- %
\begin{42console} % PYTHON ENV SERAIT MIEUX
	languages = {
    	'Python': 'Guido van Rossum',
    	'Ruby': 'Yukihiro Matsumoto',
    	'PHP': 'Rasmus Lerdorf',
    	}
\end{42console}

Using the \texttt{languages} dictionary above, a similar exercise:
Using the texttt{languages} dictionary above, write a program similar to the previous kata:

\begin{42console}
	$> python kata01.py
	Python was created by Guido van Rossum
	Ruby was created by Yukihiro Matsumoto
	PHP was created by Rasmus Lerdorf
\end{42console}

% ================================= %
\section*{kata02}
% --------------------------------- %
\begin{42console} % PYTHON ENV SERAIT MIEUX
	t = (3,30,2019,9,25)
\end{42console}

Given the tuple above, whose values stand for: \texttt{(hour, minutes, year, month, day)},
write a program that displays it in the following format:

\begin{42console}
	$> python kata02.py
	09/25/2019 03:30
\end{42console}

% ================================= %
\section*{kata03}
% --------------------------------- %
\begin{42console} % PYTHON ENV SERAIT MIEUX
	phrase = "The right format"
\end{42console}

Write a program to display the string above right-aligned with '-'
padding and a total length of 42 characters:

\begin{42console}
	$> python kata03.py | cat -e
	--------------------------The right format%
	$> python kata03.py | wc -c
    42
\end{42console}

\begin{42console}
$> python kata03.py | cat -e
--------------------------The right format%
$> python kata03.py | wc -c
42
\end{end}

% ================================= %
\section*{kata04}
% --------------------------------- %
\begin{42console} % PYTHON ENV SERAIT MIEUX
	t = ( 0, 4, 132.42222, 10000, 12345.67)
\end{42console}

Given the tuple above, write a program which displays a formatted string:

\begin{42console}
	$> python kata04.py
	module_00, ex_04 : 132.42, 1.00e+04, 1.23e+04
\end{42console}

% ===========================(fin ex 05)         %
% ============================================== %

\newpage

% ============================================== %
% ===========================(start ex 06)       %
\chapter{Exercise 06}
\extitle{A recipe}
\turnindir{ex06}
\exnumber{06}
\exfiles{recipe.py}
\exforbidden{None}
\makeheaderfilesforbidden

It is time to discover Python dictionaries. Dictionaries are collections
that contain mappings of unique keys to values.

\hint{Check what is a nested dictionary in Python.}

% ================================= %
% \section*{}
% --------------------------------- %
First, you have to create a cookbook dictionary called \texttt{cookbook}.

\texttt{cookbook} will store 3 recipes:
\begin{itemize}
	\item sandwich
	\item cake
	\item salad
\end{itemize}

Each recipe will store 3 values:
\begin{itemize}
	\item ingredients: a \textbf{list} of ingredients
	\item meal: type of meal
	\item prep\_time: preparation time in minutes
\end{itemize}

Sandwich's ingredients are \textit{ham}, \textit{bread}, \textit{cheese} and \textit{tomatoes}.
It is a \textit{lunch} and it takes $10$ minutes of preparation.
Cake's ingredients are \textit{flour}, \textit{sugar} and \textit{eggs}.
It is a \textit{dessert} and it takes $60$ minutes of preparation.
Salad's ingredients are \textit{avocado}, \textit{arugula}, \textit{tomatoes} and \textit{spinach}.
It is a \textit{lunch} and it takes $15$ minutes of preparation.

\begin{enumerate}
	\item Get to know dictionaries. In the first place, try to print only the \texttt{keys} of the dictionary. Then only the \texttt{values}. And to conclude, all the \texttt{items}.
	\item Write a function to print a recipe from \texttt{cookbook} dictionary. The function parameter will be the name of the recipe.
	\item Write a function to delete a recipe from \texttt{cookbook} dictionary. The function parameter will be the name of the recipe.
	\item Write a function to add a new recipe to \texttt{cookbook} with its ingredients, its meal type and its preparation time. The function parameters will be the name of recipe, the ingredients, the meal type and preparation time.
	\item Write a function to print all recipe names from \texttt{cookbook} dictionary. Think about formatting the output.
	\item Last but not least, make a program using the four functions you just created.
\end{enumerate}

The program will prompt the user to make a choice between printing the cookbook, printing only one recipe, adding a recipe, deleting a recipe or quitting the cookbook.


It could look like the example below but feel free to organize it the way you want to:

\begin{42console}
	$> python recipe.py
	Please select an option by typing the corresponding number:
	1: Add a recipe
	2: Delete a recipe
	3: Print a recipe
	4: Print the cookbook
	5: Quit

	>> 3
	Please enter the recipe's name to get its details:

	>> cake
	Recipe for cake:
	Ingredients list: ['flour', 'sugar', 'eggs']
	To be eaten for dessert.
	Takes 60 minutes of cooking.
\end{42console}

Your program must continue running until the user exits it (option 5):

\begin{42console}
	$> python recipe.py
	Please select an option by typing the corresponding number:
	1: Add a recipe
	2: Delete a recipe
	3: Print a recipe
	4: Print the cookbook
	5: Quit

	>> 5
	Cookbook closed.
	$>
\end{42console}

The program will also continue running if the user enters a wrong value.
It will prompt the user again until the value is correct:

\begin{42console}
	$> python recipe.py
	Please select an option by typing the corresponding number:
	1: Add a recipe
	2: Delete a recipe
	3: Print a recipe
	4: Print the cookbook
	5: Quit

	>> test
	This option does not exist, please type the corresponding number.
	To exit, enter 5.
\end{42console}


% ===========================(fin ex 06)         %
% ============================================== %

\newpage

% ============================================== %
% ===========================(start ex 07)       %
\chapter{Exercise 07}
\extitle{Shorter, faster, pythonest}
\turnindir{ex07}
\exnumber{07}
\exfiles{filterwords.py}
\exforbidden{filter}
\makeheaderfilesforbidden

Using list comprehensions, you have to make a program that removes all
the words in a string that are shorter than or equal to n letters, and
returns the filtered list with no punctuation.
The program will accept only two parameters: a string, and an integer n.
The string parameter can contain number: "this string has 1 number.".

% ================================= %
\section*{Examples}
% --------------------------------- %

\begin{42console}
	$> python filterwords.py "Hello, my friend" 3
	['Hello', 'friend']
	$>
	$>  python filterwords.py "A robot must protect its own existence as long as such protection does not conflict with the First or Second Law" 6
	['protect', 'existence', 'protection', 'conflict']
	$>
	$> python filterwords.py Hello World
	ERROR
	$>
	$> python filterwords.py 300 3
	[]
\end{42console}

% ===========================(fin ex 07)         %
% ============================================== %

\newpage

% ============================================== %
% ===========================(start ex 08)       %
\chapter{Exercise 08}
\extitle{S.O.S}
\turnindir{ex08}
\exnumber{08}
\exfiles{sos.py}
\exforbidden{None}
\makeheaderfilesforbidden

You have to make a function which encodes strings into Morse code.
All alphanumeric characters are accepted by the encoder.

% ================================= %
\section*{Examples}
% --------------------------------- %

\begin{42console}
	$> python sos.py "SOS"
	... --- ...
	$> python sos.py
	$> python sos.py "HELLO / WORLD"
	ERROR
	$> python sos.py "96 BOULEVARD" "Bessiere"
	----. -.... / -... --- ..- .-.. . ...- .- .-. -.. / -... . ... ... .. . .-. .
\end{42console}


\hint{
	\url{https://morsecode.world/international/morse2.html}
}
% ===========================(fin ex 08)         %
% ============================================== %

\newpage

% ============================================== %
% ===========================(start ex 09)       %
\chapter{Exercise 09}
\extitle{Secret number}
\turnindir{ex09}
\exnumber{09}
\exfiles{guess.py}
\exforbidden{None}
\makeheaderfilesforbidden

You have to make a program that will be an interactive guessing game.
It will ask the user to guess a number between $1$ and $99$.
The program will tell the user if their input is too high or too low.
The game ends when the user finds out the secret number or types \texttt{exit}.
You will import the \texttt{random} module with the \texttt{randint} function to get a random number.
You have to count the number of trials and print that number when the user wins.

% ================================= %
\section*{Examples}
% --------------------------------- %

\begin{42console}
	$> python guess.py
	This is an interactive guessing game!
	You have to enter a number between 1 and 99 to find out the secret number.
	Type 'exit' to end the game.
	Good luck!

	What's your guess between 1 and 99?
	>> 54
	Too high!
	What's your guess between 1 and 99?
	>> 34
	Too low!
	What's your guess between 1 and 99?
	>> 45
	Too high!
	What's your guess between 1 and 99?
	>> A
	That's not a number.
	What's your guess between 1 and 99?
	>> 43
	Congratulations, you've got it!
	You won in 5 attempts!
\end{42console}

If the user discovers the secret number on the first try, tell them.
If the secret number is 42, make a reference to Douglas Adams.

\begin{42console}
	$> python guess.py
	This is an interactive guessing game!
	You have to enter a number between 1 and 99 to find out the secret number.
	Type 'exit' to end the game.
	Good luck!

	What's your guess between 1 and 99?
	>> 42
	The answer to the ultimate question of life, the universe and everything is 42.
	Congratulations! You got it on your first try!
\end{42console}


Other example:

\begin{42console}
	$> python guess.py
	This is an interactive guessing game!
	You have to enter a number between 1 and 99 to find out the secret number.
	Type 'exit' to end the game.
	Good luck!

	What's your guess between 1 and 99?
	>> exit
	Goodbye!
\end{42console}

% ===========================(fin ex 09)         %
% ============================================== %

\newpage

% ============================================== %
% ===========================(start ex 10)       %
\chapter{Exercise 10}
\extitle{Loading bar!}
\turnindir{ex10}
\exnumber{10}
\exfiles{loading.py}
\exforbidden{None}
\makeheaderfilesforbidden


You are about to discover the \texttt{yield} operator!

So let's create a function called \texttt{ft\_progress(lst)}.

The function will display the progress of a \texttt{for} loop.


% ================================= %
\section*{Examples}
% --------------------------------- %
\begin{42console}% SERAIT MIEUX AVEC UN ENVIRONNEMENT PYTHON
	listy = range(1000)
	ret = 0
	for elem in ft_progress(listy):
	    ret += (elem + 3) % 5
	    sleep(0.01)
	print()
	print(ret)
\end{42console}

\begin{42console}
	$> python loading.py
	ETA: 8.67s [ 23%][=====>                  ] 233/1000 | elapsed time 2.33s
	...
	2000
\end{42console}

\begin{42console}% SERAIT MIEUX AVEC UN ENVIRONNEMENT PYTHON
	listy = range(3333)
	ret = 0
	for elem in ft_progress(listy):
		ret += elem
		sleep(0.005)
	print()
	print(ret)
\end{42console}

\begin{42console}
	$> python loading.py
	ETA: 14.67s [  9%][=>                      ] 327/3333 | elapsed time 1.33s
	...
	5552778
\end{42console}


% ===========================(fin ex 10)         %
% ============================================== %
\newpage
% ================================= %
\section*{Contact}
% --------------------------------- %
You can contact 42AI association by email: contact@42ai.fr\\
You can join the association on \href{https://join.slack.com/t/42-ai/shared_invite/zt-ebccw5r7-YPkDM6xOiYRPjqJXkrKgcA}{42AI slack}
and/or apply to \href{https://forms.gle/VAFuREWaLmaqZw2D8}{one of the association teams}.

% ================================= %
\section*{Acknowledgements}
% --------------------------------- %
The modules Python \& ML is the result of a collective work, we would like to thanks:
\begin{itemize}
  \item Maxime Choulika (cmaxime),
  \item Pierre Peigné (ppeigne),
  \item Matthieu David (mdavid).
\end{itemize}
who supervised the creation, the enhancement and this present transcription.

\begin{itemize}
    \item Amric Trudel (amric@42ai.fr)
    \item Baptiste Lefeuvre (blefeuvr@student.42.fr)
    \item Mathilde Boivin (mboivin@student.42.fr)
    \item Tristan Duquesne (tduquesn@student.42.fr)
    \item Quentin Feuillade Montixi (qfeuilla@student.42.fr)
\end{itemize}
for your investment for the creation and development of these modules.

\begin{itemize}
    \item Barthélémy Leveque (bleveque@student.42.fr)
    \item Remy Oster (roster@student.42.fr)
    \item Quentin Bragard (qbragard@student.42.fr)
    \item Marie Dufourq (madufour@student.42.fr)
    \item Adrien Vardon (advardon@student.42.fr)
\end{itemize}
who betatest the first version of the modules of Machine Learning.
\vfill
\doclicenseThis

\end{document}
