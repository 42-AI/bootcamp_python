%******************************************************************************%
%                                                                              %
%                        Common Instructions                                   %
%                          for Python Projects                                 %
%                                                                              %
%******************************************************************************%

\chapter{Common Instructions}
\begin{itemize}
  \item The version of Python recommended to use is 3.7, you can
  check the version of Python with the following command: \texttt{python -V}
  
  \item The norm: during this bootcamp, it is recommended to follow the
  \href{https://www.python.org/dev/peps/pep-0008/}{PEP 8 standards}, though it is not mandatory.
  You can install \href{https://pypi.org/project/pycodestyle}{pycodestyle} which
  is a tool to check your Python code.
  \item The function \texttt{eval} is never allowed.
  \item The exercises are ordered from the easiest to the hardest.
  \item Your exercises are going to be evaluated by someone else,
  so make sure that your variable names and function names are appropriate and civil.
  \item Your manual is the internet.
  
  \item If you are a student from 42, you can access our Discord server 
  on \href{https://discord.com/channels/887850395697807362/887850396314398720}{42 student's associations portal} and ask your
  questions to your peers in the dedicated Bootcamp channel. 

  \item You can also ask questions in the \texttt{\#bootcamps} channel on Slack at \href{https://42-ai.slack.com}{42AI}
  or \href{42born2code.slack.com}{42born2code}.
  
  \item If you find any issue or mistake in the subject please create an issue on \href{https://github.com/42-AI/bootcamp_python/issues}{42AI repository on Github}.
  
  \item We encourage you to create test programs for your
  project even though this work \textbf{won't have to be
  submitted and won't be graded}. It will give you a chance
  to easily test your work and your peers’ work. You will find
  those tests especially useful during your defence. Indeed,
  during defence, you are free to use your tests and/or the
  tests of the peer you are evaluating.
  
  \item Submit your work to your assigned git repository. Only the work in the
  git repository will be graded. If Deepthought is assigned to grade your
  work, it will be run after your peer-evaluations.
  If an error happens in any section of your work during Deepthought's grading,
  the evaluation will stop.
\end{itemize}