\chapter{Exercise 00}
\extitle{\$PATH}
\turnindir{ex00}
\exnumber{00}
\exfiles{answers.txt, requirements.txt}
\exforbidden{None}
\makeheaderfilesforbidden\\
\emph{The first thing you need to do is to install Python.}
\\\\
Most modern Unix-based systems have a \texttt{python} interpreter installed by default,
but its version might be lower/higher than the one used for these modules.
It is also possible that the default \texttt{python} command uses a version 2.x (for legacy reasons).
This is obviously very confusing for a new developper.
\\

\begin{42console}
$> python -V
$> python3 -V
\end{42console}
\hfill \break
To deal with those versions issues we will use \texttt{conda}. This program allows you to manage your Python packages and several working environments. 
\\\\
\emph{Note: the actual requirement is to use a Python 3.7.X version. You are free to use a different program/utilities to achieve this goal. At your own risk.}
\pagebreak

% ===========================(Conda Manual)       %
\section*{Conda manual installation}
\emph{Go to the next section for an automated installation.}
\\\\
We recommend the following path for your \texttt{conda} folder.
\begin{42console}
$> MYPATH="/goinfre/$USER/miniconda3"
\end{42console}

% ---------------------------------- %
\subsection*{1. Download \& Install conda}

\begin{42console}
# For MAC
$> curl -LO "https://repo.anaconda.com/miniconda/Miniconda3-latest-MacOSX-x86_64.sh"
$> sh Miniconda3-latest-MacOSX-x86_64.sh -b -p $MYPATH

# For Linux
$> curl -LO "https://repo.anaconda.com/miniconda/Miniconda3-latest-Linux-x86_64.sh"
$> sh Miniconda3-latest-Linux-x86_64.sh -b -p $MYPATH
\end{42console}



% ---------------------------------- %
\subsection*{2. Initial configuration of conda}
\begin{42console}
# For zsh
$> $MYPATH/bin/conda init zsh
$> $MYPATH/bin/conda config --set auto_activate_base false
$> source ~/.zshrc

# For bash
$> $MYPATH/bin/conda init bash
$> $MYPATH/bin/conda config --set auto_activate_base false
$> source ~/.bash_profile
\end{42console}



% ---------------------------------- %
\subsection*{3. Create a dedicated environment for 42AI !}
\begin{42console}
$> conda create --name 42AI-$USER python=3.7 jupyter pandas pycodestyle numpy
\end{42console}



% ---------------------------------- %
\subsection*{4. Check your 42AI Python environment}
\begin{42console}
$> conda info --envs
$> conda activate 42AI-$USER
$> which python
$> python -V
$> python -c "print('Hello World!')"
\end{42console}



% ---------------------------------- %
\subsection*{Help !}
\begin{itemize}
	\item \textbf{I have lost my miniconda3 folder !} Repeat step 1 and 3.
	\item \textbf{I have lost my home directory !} Repeat step 2.
\end{itemize}
\pagebreak



% ===========================(Conda Auto)       %
\section*{Conda automated installation}

Copy the following script into a file, launch it and follow the instructions. It downloads and installs miniconda in a \texttt{/goinfre} subfolder and creates a \texttt{python} environment in \texttt{conda}.


\begin{42console}
#!/bin/bash

function which_dl {
	# If operating system name contains Darwnin: MacOS. Else Linux
	if uname -s | grep -iqF Darwin; then
		echo "Miniconda3-latest-MacOSX-x86_64.sh"
	else
		echo "Miniconda3-latest-Linux-x86_64.sh"
	fi
}

function which_shell {
	# if $SHELL contains zsh, zsh. Else Bash
	if echo $SHELL | grep -iqF zsh; then
		echo "zsh"
	else
		echo "bash"
	fi
}

function when_conda_exist {
	# check and install 42AI environement
	printf "Checking 42AI-$USER environment: "
	if conda info --envs | grep -iqF 42AI-$USER; then
		printf "\e[33mDONE\e[0m\n"
	else
		printf "\e[31mKO\e[0m\n"
		printf "\e[33mCreating 42AI environnment:\e[0m\n"
		conda update -n base -c defaults conda -y
		conda create --name 42AI-$USER python=3.7 jupyter numpy pandas pycodestyle -y
	fi
}

function set_conda {
	MINICONDA_PATH="/goinfre/$USER/miniconda3"
	CONDA=$MINICONDA_PATH"/bin/conda"
	PYTHON_PATH=$(which python)
	REQUIREMENTS="jupyter numpy pandas pycodestyle"
	SCRIPT=$(which_dl)
	MY_SHELL=$(which_shell)
	DL_LINK="https://repo.anaconda.com/miniconda/"$SCRIPT
	DL_LOCATION="/tmp/"

	printf "Checking conda: "
	TEST=$(conda -h 2>/dev/null)
	if [ $? == 0 ] ; then
		printf "\e[32mOK\e[0m\n"
		when_conda_exist
		return
	fi
	printf "\e[31mKO\e[0m\n"
	if [ ! -f $DL_LOCATION$SCRIPT ]; then
		printf "\e[33mDonwloading installer:\e[0m\n"
		cd $DL_LOCATION
		curl -LO $DL_LINK
		cd -
	fi
	printf "\e[33mInstalling conda:\e[0m\n"
	sh $DL_LOCATION$SCRIPT -b -p $MINICONDA_PATH
	printf "\e[33mConda initial setup:\e[0m\n"
	$CONDA init $MY_SHELL
	$CONDA config --set auto_activate_base false

	printf "\e[33mCreating 42AI-$USER environnment:\e[0m\n"
	$CONDA update -n base -c defaults conda -y
	$CONDA create --name 42AI-$USER python=3.7 jupyter numpy pandas pycodestyle -y
	printf "\e[33mLaunch the following command or restart your shell:\e[0m\n"
	if [ $MY_SHELL == "zsh" ]; then
		printf "\tsource ~/.zshrc\n"
	else
		printf "\tsource ~/.bash_profile\n"
	fi
}

set_conda
\end{42console}
\hfill \break
Don't forget to check your 42AI Python environment !
\begin{42console}
conda info --envs
conda activate 42AI-$USER
which python
python -V
python -c "print('Hello World!')"
\end{42console}

% =================================(Question) %
\section*{(Finally) getting started}

Now that your setup is ready to run, here are a few questions that need to be solved using \texttt{python}, \texttt{pip} or \texttt{conda}. Save your answers in a file \texttt{answers.txt} (one answer per line and per question), and check them with your peers.\\\\
Find the commands to:
\begin{itemize}
	\item Output a list of installed packages and their versions.
	\item Show the package metadata of \texttt{numpy}.
	\item Remove the package \texttt{numpy}.
	\item (Re)install the package \texttt{numpy}.
	\item Freeze your \texttt{python} packages and their versions in a \texttt{requirements.txt} file you have to turn-in.
\end{itemize}


