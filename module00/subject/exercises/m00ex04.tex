\chapter{Exercise 04}
\extitle{Elementary}
\turnindir{ex04}
\exnumber{04}
\exfiles{operations.py}
\exforbidden{None}
\makeheaderfilesforbidden\\
Write a program that takes two integers A and B as arguments and prints the result of the following operations:
\newline
\begin{42console}
Sum:         	A+B
Difference: 	A-B
Product:     	A*B
Quotient:    	A/B
Remainder:   	A%B
\end{42console}

\begin{itemize}
	\item If more or less than two arguments are provided or if one of the arguments is not an integer, print an error message.
	\item If no argument is provided, do nothing or print an usage.
	\item If an operation is impossible, print an error message instead of a numerical result.
\end{itemize}

% ================================= %
\section*{Examples}
\begin{42console}
$> python3 operations.py 10 3
Sum:         13
Difference:  7
Product:     30
Quotient:    3.3333...
Remainder:   1
$>
$> python3 operations.py 42 10
Sum:         52
Difference:  32
Product:     420
Quotient:    4.2
Remainder:   2
$>
$> python3 operations.py 1 0
Sum:         1
Difference:  1
Product:     0
Quotient:    ERROR (division by zero)
Remainder:   ERROR (modulo by zero)
$>
$> python3 operations.py
Usage: python operations.py <number1> <number2>
Example:
	python operations.py 10 3
$>
$> python3 operations.py 12 10 5
AssertionError: too many arguments
$>
$> python3 operations.py "one" "two"
AssertionError: only integers
$>
\end{42console}

\hint{
	No bonus point to be gained from handling decimal point or scientific notation. Keep it simple.
}