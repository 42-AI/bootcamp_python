\chapter{Exercise 06}
\extitle{A recipe}
\turnindir{ex06}
\exnumber{06}
\exfiles{recipe.py}
\exforbidden{None}
\makeheaderfilesforbidden

% ================================= %
\section*{Part 1: Nested Dictionaries}

Create a dictionary called \texttt{cookbook}. You will use this \texttt{cookbook} to store recipes.
\\\\
A recipe is a \textbf{dictionary} that stores (at least) 3 key-value pairs:
\begin{itemize}
	\item ''ingredients": a \textbf{list of strings} containing the list of ingredients
	\item "meal": a \textbf{string} representing the type of meal
	\item "prep\_time": a \textbf{non-negative integer} representing a preparation time in minutes
\end{itemize}
In the \texttt{cookbook}, the  \textbf{key} to a recipe is the recipe's name.
\\
Initialize your \texttt{cookbook} with 3 recipes:
\begin{itemize}
	\item The Sandwich's ingredients are \textit{ham}, \textit{bread}, \textit{cheese} and \textit{tomatoes}.
It is a \textit{lunch} and it takes $10$ minutes of preparation.
	\item The Cake's ingredients are \textit{flour}, \textit{sugar} and \textit{eggs}.
It is a \textit{dessert} and it takes $60$ minutes of preparation.
	\item The Salad's ingredients are \textit{avocado}, \textit{arugula}, \textit{tomatoes} and \textit{spinach}.
It is a \textit{lunch} and it takes $15$ minutes of preparation.
\end{itemize}

% ================================= %
\newpage
\section*{Part 2: A Handful of Helpful Functions}

Create a series of useful functions to handle your \texttt{cookbook}:

\begin{enumerate}
	\item A function that prints all recipe names.
	\item A function that takes a recipe name and prints its details.
	\item A function that takes a recipe name and delete it.
	\item A function that adds a recipe from user input. You will need a name, a list of ingredients, a meal type and a preparation time.
\end{enumerate}
\subsection*{Input example}
\begin{42console}
>>> Enter a name:
chips
>>> Enter ingredients:
potatoes
oil
salt

>>> Enter a meal type:
lunch
>>> Enter a preparation time:
15
\end{42console}

% ================================= %
\section*{Part 3: A command line executable !}

Create a program that uses your \texttt{cookbook} and your functions.\\
\newline
The program will prompt the user to make a choice between printing the cookbook's content,
 printing one recipe, adding a recipe, deleting a recipe or quitting the cookbook.\\
\newline
Your program will continue to prompt the user until the user decides to quit it.\\
\newline
The program cannot crash if a wrong value is entered: you must handle the error and ask for another prompt.\\

\begin{42console}
$> python3 recipe.py
Welcome to the Python Cookbook !
List of available options:
   1: Add a recipe
   2: Delete a recipe
   3: Print a recipe
   4: Print the cookbook
   5: Quit

Please select an option:
>> 3

Please enter a recipe name to get its details:
>> cake

Recipe for cake:
   Ingredients list: ['flour', 'sugar', 'eggs']
   To be eaten for dessert.
   Takes 60 minutes of cooking.

Please select an option:
>> Hello

Sorry, this option does not exist.
List of available options:
   1: Add a recipe
   2: Delete a recipe
   3: Print a recipe
   4: Print the cookbook
   5: Quit

Please select an option:
>> 5

Cookbook closed. Goodbye !
$>
\end{42console}