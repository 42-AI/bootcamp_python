\chapter{Exercise 03}
\extitle{Functional file}
\turnindir{ex03}
\exnumber{03}
\exfiles{count.py}
\exforbidden{None}
\makeheaderfilesforbidden
\section*{Part 1. text\_analyzer}
Create a function called \texttt{text\_analyzer} that takes a single string argument
and displays the total number of printable characters, and respectively : the number of upper-case characters, lower-case characters, punctuation characters and spaces.
\begin{itemize}
	\item If \texttt{None} or nothing is provided, the user is prompted to provide a string. 
	\item If the argument is not a string, print an error message.
	\item This function must have a \texttt{docstring} explaning its behavior.
\end{itemize}
Test your function within the \texttt{python} console

% --------------------------------- %
\subsection*{Examples}
\begin{42console}
$> python3
>>> from count import text_analyzer
>>> text_analyzer("Python 2.0, released 2000, introduced
features like List comprehensions and a garbage collection
system capable of collecting reference cycles.")
The text contains 143 printable character(s):
- 2 upper letter(s)
- 113 lower letter(s)
- 4 punctuation mark(s)
- 18 space(s)
>>> text_analyzer("Python is an interpreted, high-level,
general-purpose programming language. Created by Guido van
Rossum and first released in 1991, Python's design philosophy
emphasizes code readability with its notable use of significant
whitespace.")
The text contains 234 printable character(s):
- 5 upper letter(s)
- 187 lower letter(s)
- 8 punctuation mark(s)
- 30 space(s)
>>> text_analyzer()
What is the text to analyze?
>> Hello World!
The text contains 12 printable character(s):
- 2 upper letter(s)
- 8 lower letter(s)
- 1 punctuation mark(s)
- 1 space(s)
>>> text_analyzer(42)
AssertionError: argument is not a string
>>> print(text_analyzer.__doc__)

    This function counts the number of upper characters, lower characters,
    punctuation and spaces in a given text.
\end{42console}

\hint{
	Python has a lot of very convenient built-in functions and methods. Do some research, you do not have to reivent the wheel !
}

% ================================= %
\section*{Part 2. \_\_name\_\_==\_\_main\_\_}

In the previous part, you wrote a function that can be used in the console or in another file when imported.
Without changing this behavior, update your file so it can also be launched as a standalone program.

\begin{itemize}
	\item If more than one argument is provided to the program, print an error message. 
	\item Otherwise, use the \texttt{text\_analyzer} function.
\end{itemize}

% --------------------------------- %
\subsection*{Examples}
\begin{42console}
$> python3 count.py 'Hello World!'
The text contains 12 character(s):
- 2 upper letter(s)
- 8 lower letter(s)
- 1 punctuation mark(s)
- 1 space(s)
$> python3
>>> from count import text_analyzer
>>> text_analyzer("Hello World!")
The text contains 12 character(s):
- 2 upper letter(s)
- 8 lower letter(s)
- 1 punctuation mark(s)
- 1 space(s)
\end{42console}