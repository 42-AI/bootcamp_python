\chapter{Exercise 05}
\extitle{The right format}
\turnindir{ex05}
\exnumber{05}
\exfiles{kata00.py, kata01.py, kata02.py, kata03.py, kata04.py}
\exforbidden{None}
\makeheaderfilesforbidden
Let's get familiar with the useful concept of \textbf{string formatting} through a kata series.\\
\newline
Each exercice will provide you with a \texttt{kata} variable. This variable can be modified to a certain extent: your program must react accordingly.
% ================================= %
\section*{kata00}
The \texttt{kata} variable is always a tuple and can only be filled with integers.
\begin{42console}
# Put this at the top of your kata00.py file
kata = (19,42,21)
\end{42console}
Write a program that displays this variable content according to the format shown below:

\begin{42console}
$> python3 kata00.py
The 3 numbers are: 19, 42, 21
$>
\end{42console}

% ================================= %
\section*{kata01}

The \texttt{kata} variable is always a dictionary and can only be filled with strings.

\begin{42console}
# Put this at the top of your kata01.py file
kata = {
    'Python': 'Guido van Rossum',
    'Ruby': 'Yukihiro Matsumoto',
    'PHP': 'Rasmus Lerdorf',
    }
\end{42console}
Write a program that displays this variable content according to the format shown below:

\begin{42console}
$> python3 kata01.py
Python was created by Guido van Rossum
Ruby was created by Yukihiro Matsumoto
PHP was created by Rasmus Lerdorf
$>
\end{42console}

% ================================= %
\section*{kata02}

The \texttt{kata} variable is always a tuple that contains 5 non-negative integers. The first integer contains up to 4 digits, the rest up to 2 digits.

\begin{42console} 
# Put this at the top of your kata02.py file
kata = (2019, 9, 25, 3, 30)
\end{42console}
Write a program that displays this variable content according to the format shown below:

\begin{42console}
$> python3 kata02.py | cat -e
09/25/2019 03:30$
$> python3 kata02.py | wc -c
17
$>
\end{42console}

% ================================= %
\section*{kata03}

The \texttt{kata} variable is always a string whose length is not higher than 42.

\begin{42console}
# Put this at the top of your kata03.py file
kata = "The right format"
\end{42console}
Write a program that displays this variable content according to the format shown below:

\begin{42console}
$> python3 kata03.py | cat -e
--------------------------The right format%
$> python3 kata03.py | wc -c
42
$>
\end{42console}

% ================================= %
\section*{kata04}

The \texttt{kata} variable is always a tuple that contains, in the following order:
\begin{itemize}
	\item 2 non-negative integers containing up to 2 digits
	\item 1 decimal
	\item 1 integer
	\item 1 decimal
\end{itemize}

\begin{42console}
# Put this at the top of your kata04.py file
kata = (0, 4, 132.42222, 10000, 12345.67)
\end{42console}
Write a program that displays this variable content according to the format shown below:

\begin{42console}
$> python3 kata04.py
module_00, ex_04 : 132.42, 1.00e+04, 1.23e+04
$> python3 kata04.py | cut -c 10,18
,:
\end{42console}