\chapter{Exercise 04}
\extitle{SpatioTemporalData}
\turnindir{ex04}
\exnumber{04}
\exfiles{FileLoader.py, SpatioTemporalData.py}
\exforbidden{None}
\makeheaderfilesforbidden


% ================================= %
\section*{Objective}
% --------------------------------- %
The goal of this exercise is to implement a class called \texttt{SpatioTemporalData}
that takes a dataset (pandas.DataFrame) as argument in its constructor
and implements two methods.
% ================================= %
\section*{Instructions}
% --------------------------------- %
This exercise uses the dataset \texttt{athlete\_events.csv}.\\
\\
Write a class called \texttt{SpatioTemporalData} that takes a dataset
(pandas.DataFrame) as argument in its constructor and implements the
following methods:
\begin{itemize}
  \item \texttt{when(location)}: takes a location as an argument and returns
        a list containing the years where games were held in the given location,  
  \item \texttt{where(date)}: takes a date as an argument and returns the location
        where the Olympics took place in the given year.
\end{itemize}

% ================================= %
\section*{Examples}
% --------------------------------- %
\begin{minted}[bgcolor=darcula-back,formatcom=\color{lightgrey},fontsize=\scriptsize]{python}
from FileLoader import FileLoader
loader = FileLoader()
data = loader.load('../data/athlete_events.csv')
# Output
Loading dataset of dimensions 271116 x 15


from SpatioTemporalData import SpatioTemporalData
sp = SpatioTemporalData(data)
sp.where(1896)
# Output
['Athina']


sp.where(2016)
# Output
['Rio de Janeiro']


sp.when('Athina')
# Output
[2004, 1906, 1896]


sp.when('Paris')
# Output
[1900, 1924]
\end{minted}