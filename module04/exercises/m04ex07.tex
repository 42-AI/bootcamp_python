\chapter{Exercise 07}
\extitle{Komparator}
\turnindir{ex07}
\exnumber{07}
\exfiles{Komparator.py, MyPlotLib.py (optional)}
\exforbidden{None}
\makeheaderfilesforbidden


% ================================= %
\section*{Objective}
% --------------------------------- %
The goal of this exercise is to introduce plotting methods among the different
libraries Pandas, Matplotlib, Seaborn or Scipy.

% ================================= %
\section*{Instructions}
% --------------------------------- %
This exercise uses the following dataset: \texttt{athlete\_events.csv}.\\
\\
Write a class called \texttt{Komparator} whose constructor takes as an argument a pandas.DataFrame which contains the dataset.\\
\\
The class must implement the following methods, which take as input two variable names:
\begin{itemize}
  \item \texttt{compare\_box\_plots(self, categorical\_var, numerical\_var)}: displays a series of box plots 
  to compare how the distribution of the numerical variable changes if we only consider 
  the subpopulation which belongs to each category. 
  There should be as many box plots as categories. 
  For example, with Sex and Height, we would compare 
  the height distributions of men vs. women with two box plots.
  \item \texttt{density(self, categorical\_var, numerical\_var)}: displays the density of the numerical variable.
  Each subpopulation should be represented by a separate curve on the graph.
  \item \texttt{compare\_histograms(self, categorical\_var, numerical\_var)}: plots the numerical variable in a separate histogram for each category.
  As an extra, you can use overlapping histograms with a color code.
\end{itemize}
BONUS: Your functions can also accept a list of numerical variables (instead of just one), and output a comparison plot for each variable in the list.
