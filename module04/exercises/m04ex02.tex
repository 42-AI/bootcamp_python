\chapter{Exercise 02}
\extitle{ProportionBySport}
\turnindir{ex02}
\exnumber{02}
\exfiles{FileLoader.py, ProportionBySport.py}
\exforbidden{None}
\makeheaderfilesforbidden

% ================================= %
\section*{Objective}
% --------------------------------- %
The goal of this exercise is to create a function displaying
the proportion of participants who played a given sport, among
the participants of a given gender.

% ================================= %
\section*{Instructions}
% --------------------------------- %
This exercise uses the dataset \texttt{athlete\_events.csv}.

Write a function \texttt{proportion\_by\_sport} that takes four arguments:
\begin{itemize}
  \item a pandas.DataFrame of the dataset,
  \item an olympic year,
  \item a sport,
  \item a gender.
\end{itemize}
The function returns a float corresponding to the proportion (percentage) of participants 
who played the given sport among the participants of the given gender.\\
\\
The function answers questions like the following : 
"What was the percentage of female basketball players among all female 
participants in the 2016 Olympics?"

\hint{
Here and later on, if needed, drop duplicate sports people to count only unique ones. Beware to call the dropping function
at the right moment and with the right parameters, in order not to omit any individual.
}

% ================================= %
\section*{Examples}
% --------------------------------- %
\begin{minted}[bgcolor=darcula-back,formatcom=\color{lightgrey},fontsize=\scriptsize]{python}
from FileLoader import FileLoader
loader = FileLoader()
data = loader.load('../data/athlete_events.csv')
# Output
Loading dataset of dimensions 271116 x 15


from ProportionBySport import proportion_by_sport
proportion_by_sport(data, 2004, 'Tennis', 'F')
# Output
0.019302325581395347
\end{minted}
\newline
We assume that we are always using valid arguments as input,
and thus do not need to handle input errors.