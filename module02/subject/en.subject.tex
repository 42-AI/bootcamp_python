% vim: set ts=4 sw=4 tw=80 noexpandtab:

\documentclass{42-en}

%******************************************************************************%
%                                                                              %
%                                   Prologue                                   %
%                                                                              %
%******************************************************************************%
\usepackage{amsmath}
\usepackage[
    type={CC},
    modifier={by-nc-sa},
    version={4.0},
]{doclicense}

%****************************************************************%
%                  Re/definition of commands                     %
%****************************************************************%

\newcommand{\ailogo}[1]{\def \@ailogo {#1}}\ailogo{assets/42ai_logo.pdf}

%%  Redefine \maketitle
\makeatletter
\def \maketitle {
  \begin{titlepage}
    \begin{center}
	%\begin{figure}[t]
	  %\includegraphics[height=8cm]{\@ailogo}
	  \includegraphics[height=8cm]{assets/42ai_logo.pdf}
	%\end{figure}
      \vskip 5em
      {\huge \@title}
      \vskip 2em
      {\LARGE \@subtitle}
      \vskip 4em
    \end{center}
    %\begin{center}
	  %\@author
    %\end{center}
	%\vskip 5em
  \vfill
  \begin{center}
    \emph{\summarytitle : \@summary}
  \end{center}
  \vspace{2cm}
  %\vskip 5em
  %\doclicenseThis
  \end{titlepage}
}
\makeatother

\makeatletter
\def \makeheaderfilesforbidden
{
  \noindent
  \begin{tabularx}{\textwidth}{|X X  X X|}
    \hline
  \multicolumn{1}{|>{\raggedright}m{1cm}|}
  {\vskip 2mm \includegraphics[height=1cm]{assets/42ai_logo.pdf}} &
  \multicolumn{2}{>{\centering}m{12cm}}{\small Exercise : \@exnumber } &
  \multicolumn{1}{ >{\raggedleft}p{1.5cm}|}
%%              {\scriptsize points : \@exscore} \\ \hline
              {} \\ \hline

  \multicolumn{4}{|>{\centering}m{15cm}|}
              {\small \@extitle} \\ \hline

  \multicolumn{4}{|>{\raggedright}m{15cm}|}
              {\small Turn-in directory : \ttfamily
                $ex\@exnumber/$ }
              \\ \hline
  \multicolumn{4}{|>{\raggedright}m{15cm}|}
              {\small Files to turn in : \ttfamily \@exfiles }
              \\ \hline

  \multicolumn{4}{|>{\raggedright}m{15cm}|}
              {\small Forbidden functions : \ttfamily \@exforbidden }
              \\ \hline

%%  \multicolumn{4}{|>{\raggedright}m{15cm}|}
%%              {\small Remarks : \ttfamily \@exnotes }
%%              \\ \hline
\end{tabularx}
%% \exnotes
\exrules
\exmake
\exauthorize{None}
\exforbidden{None}
\extitle{}
\exnumber{}
}
\makeatother

%%  Syntactic highlights
\makeatletter
\newenvironment{pythoncode}{%
  \VerbatimEnvironment
  \usemintedstyle{emacs}
  \minted@resetoptions
  \setkeys{minted@opt}{bgcolor=black,formatcom=\color{lightgrey},fontsize=\scriptsize}
  \begin{figure}[ht!]
    \centering
    \begin{minipage}{16cm}
      \begin{VerbatimOut}{\jobname.pyg}}
{%[
      \end{VerbatimOut}
      \minted@pygmentize{c}
      \DeleteFile{\jobname.pyg}
    \end{minipage}
\end{figure}}
\makeatother

\usemintedstyle{native}

\begin{document}

% =============================================================================%
%                     =====================================                    %

\title{Python \& ML - Module 02}
\subtitle{Basics 3}
\author{
  Maxime Choulika (cmaxime), Pierre Peigné (ppeigne), Matthieu David (mdavid)
}

\summary
{
  Let's continue practicing with more advanced Python programming exercises.
  Destination: Decorators, lambda, context manager and build package.
}

\maketitle
%******************************************************************************%
%                                                                              %
%                        Common Instructions                                   %
%                          for Python Projects                                 %
%                                                                              %
%******************************************************************************%

\chapter{Common Instructions}
\begin{itemize}
  \item The version of Python recommended to use is 3.7, you can
  check the version of Python with the following command: \texttt{python -V}
  
  \item The norm: during this piscine, it is recommended to follow the
  \href{https://www.python.org/dev/peps/pep-0008/}{PEP 8 standards}, though it is not mandatory.
  You can install \href{https://pypi.org/project/pycodestyle}{pycodestyle} which
  is a tool to check your Python code.
  \item The function \texttt{eval} is never allowed.
  \item The exercises are ordered from the easiest to the hardest.
  \item Your exercises are going to be evaluated by someone else,
  so make sure that your variable names and function names are appropriate and civil.
  \item Your manual is the internet.
  
  \item You can also ask questions in the \texttt{\#bootcamps} channel in the \href{https://42-ai.slack.com}{42AI}
  or \href{42born2code.slack.com}{42born2code}.
  
  \item If you find any issue or mistake in the subject please create an issue on \href{https://github.com/42-AI/bootcamp_python/issues}{42AI repository on Github}.
  
  \item We encourage you to create test programs for your
  project even though this work \textbf{won't have to be
  submitted and won't be graded}. It will give you a chance
  to easily test your work and your peers’ work. You will find
  those tests especially useful during your defence. Indeed,
  during defence, you are free to use your tests and/or the
  tests of the peer you are evaluating.
  
  \item Submit your work to your assigned git repository. Only the work in the
  git repository will be graded. If Deepthought is assigned to grade your
  work, it will be run after your peer-evaluations.
  If an error happens in any section of your work during Deepthought's grading,
  the evaluation will stop.
\end{itemize}
\newpage
\tableofcontents
\startexercices

%                     =====================================                    %
% =============================================================================%


%******************************************************************************%
%                                                                              %
%                                   Exercises                                  %
%                                                                              %
%******************************************************************************%

% ============================================== %
% ===========================(start ex 00)       %
\chapter{Exercise 00}
\extitle{Map, filter, reduce}
\turnindir{ex00}
\exnumber{00}
\exfiles{ft\_map.py, ft\_filter.py, ft\_reduce.py}
\exforbidden{map, filter, reduce}
\makeheaderfilesforbidden

% ================================== %
\section*{Objective}
% ---------------------------------- %
The goal of the exercise is to work on the built-in functions \texttt{map},
\texttt{filter} and \texttt{reduce}.

% ================================== %
\section*{Instructions}
% ---------------------------------- %
Implement the functions \texttt{ft\_map}, \texttt{ft\_filter} and \texttt{ft\_reduce}.
Take the time to understand the use cases of these two built-in functions
(\texttt{map} and \texttt{filter}) and the function \texttt{reduce} in functools module.
You are not expected to code specific classes to create \texttt{ft\_map},
\texttt{ft\_filter} or \texttt{ft\_reduce} objects, take a closer look
to the examples to know what to do.

Here the signatures of the functions:

\begin{minted}[bgcolor=darcula-back,formatcom=\color{lightgrey},fontsize=\scriptsize]{python}
def ft_map(function_to_apply, iterable):
	"""Map the function to all elements of the iterable.
	Args:
	  function_to_apply: a function taking an iterable.
	  iterable: an iterable object (list, tuple, iterator).
	Return:
	  An iterable.
	  None if the iterable can not be used by the function.
	"""
  # ... Your code here ...

def ft_filter(function_to_apply, iterable):
	"""Filter the result of function apply to all elements of the iterable.
	Args:
	  function_to_apply: a function taking an iterable.
	  iterable: an iterable object (list, tuple, iterator).
	Return:
	  An iterable.
	  None if the iterable can not be used by the function.
	"""
  # ... Your code here ...

def ft_reduce(function_to_apply, iterable):
	"""Apply function of two arguments cumulatively.
	Args:
	  function_to_apply: a function taking an iterable.
	  iterable: an iterable object (list, tuple, iterator).
	Return:
	  A value, of same type of elements in the iterable parameter.
	  None if the iterable can not be used by the function.
	"""
  # ... Your code here ...
\end{minted}

% ================================== %
\section*{Examples}
% ---------------------------------- %

\begin{minted}[bgcolor=darcula-back,formatcom=\color{lightgrey},fontsize=\scriptsize]{python}
# Example 1:
x = [1, 2, 3, 4, 5]
ft_map(lambda dum: dum + 1, x)
# Output:
<generator object ft_map at 0x7f708faab7b0> # The adress will be different

list(ft_map(lambda t: t + 1, x))
# Output:
[2, 3, 4, 5, 6]

# Example 2:
ft_filter(lambda dum: not (dum % 2), x)
# Output:
<generator object ft_filter at 0x7f709c777d00> # The adress will be different

list(ft_filter(lambda dum: not (dum % 2), x))
# Output:
[2, 4]

# Example 3:
lst = ['H', 'e', 'l', 'l', 'o', ' ', 'w', 'o', 'r', 'l', 'd']
ft_reduce(lambda u, v: u + v, lst)
# Output:
"Hello world"
\end{minted}

You are expected to produce the raise of exception for the functions similar to exceptions of
\texttt{map}, \texttt{filter} and \texttt{reduce} when wrong parameters are given (but no need
to reproduce the exact the same exception messages).

% ===========================(fin ex 00)         %
% ============================================== %
\newpage

% ============================================== %
% ===========================(start ex 01)       %
\chapter{Exercise 01}
\extitle{args and kwargs?}
\turnindir{ex01}
\exnumber{01}
\exfiles{main.py}
\exforbidden{None}
\makeheaderfilesforbidden

% ================================= %
\section*{Objective}
% --------------------------------- %
The goal of the exercise is to discover and manipulate \texttt{*args} and \texttt{**kwargs} arguments.

% ================================= %
\section*{Instructions}
% --------------------------------- %
In this exercise you have to implement a function named \texttt{what\_are\_the\_vars}
which returns an instance of class ObjectC.\\
ObjectC attributes are set via the parameters received during the instanciation.
You will have to modify the 'instance' \texttt{ObjectC}, \textbf{NOT} the class.\\
You should take a look to \texttt{getattr}, \texttt{setattr} built-in functions.

\begin{minted}[bgcolor=darcula-back,formatcom=\color{lightgrey},fontsize=\scriptsize]{python}
def what_are_the_vars(...):
	"""
	...
	"""
	# ... Your code here ...

class ObjectC(object):
	def __init__(self):
		# ... Your code here ...

def doom_printer(obj):
	if obj is None:
		print("ERROR")
		print("end")
		return
	for attr in dir(obj):
		if attr[0] != '_':
			value = getattr(obj, attr)
			print("{}: {}".format(attr, value))
	print("end")

if __name__ == "__main__":
	obj = what_are_the_vars(7)
	doom_printer(obj)
  obj = what_are_the_vars(None, [])
	doom_printer(obj)
	obj = what_are_the_vars("ft_lol", "Hi")
	doom_printer(obj)
	obj = what_are_the_vars()
	doom_printer(obj)
	obj = what_are_the_vars(12, "Yes", [0, 0, 0], a=10, hello="world")
	doom_printer(obj)
	obj = what_are_the_vars(42, a=10, var_0="world")
	doom_printer(obj)
  obj = what_are_the_vars(42, "Yes", a=10, var_2="world")
	doom_printer(obj)
\end{minted}

% ================================= %
\section*{Examples}
% --------------------------------- %

\begin{minted}[bgcolor=darcula-back,formatcom=\color{lightgrey},fontsize=\scriptsize]{python}
$> python main.py
var_0: 7
end
var_0: None
var_1: []
end
var_0: ft_lol
var_1: Hi
end
end
a: 10
hello: world
var_0: 12
var_1: Yes
var_2: [0, 0, 0]
end
ERROR
end
a: 10
var_0: 12
var_1: Yes
var_2: world
end
\end{minted}

% ===========================(fin ex 01)         %
% ============================================== %

\newpage

% ============================================== %
% ===========================(start ex 02)       %
\chapter{Exercise 02}
\extitle{The logger}
\turnindir{ex02}
\exnumber{02}
\exfiles{logger.py}
\exforbidden{None}
\makeheaderfilesforbidden

% ================================= %
\section*{Objective}
% --------------------------------- %
In this exercise, you will learn about decorators and we are not talking about 
the decoration of your room.
The \texttt{@log} will write info about the decorated function in a
\texttt{machine.log} file.

% ================================= %
\section*{Instructions}
% --------------------------------- %
You have to create the log decorator in the same file.
Pay attention to all the different actions logged at the call of
each methods. You may notice the username from environment
variable is written to the log file.


\begin{minted}[bgcolor=darcula-back,formatcom=\color{lightgrey},fontsize=\scriptsize]{python}
import time
from random import randint
import os

#... your definition of log decorator...

class CoffeeMachine():

	water_level = 100

	@log
	def start_machine(self):
	  if self.water_level > 20:
		  return True
	  else:
		  print("Please add water!")
		  return False
	
	@log
	def boil_water(self):
		return "boiling..."
	
	@log
	def make_coffee(self):
		if self.start_machine():
			for _ in range(20):
				time.sleep(0.1)
				self.water_level -= 1
			print(self.boil_water())
			print("Coffee is ready!")
	
	@log
	def add_water(self, water_level):
		time.sleep(randint(1, 5))
		self.water_level += water_level
		print("Blub blub blub...")


if __name__ == "__main__":
	
	machine = CoffeeMachine()
	for i in range(0, 5):
		machine.make_coffee()

	machine.make_coffee()
	machine.add_water(70)
\end{minted}

% ================================= %
\section*{Examples}
% --------------------------------- %
\begin{42console}
  $> python logger.py
  boiling...
  Coffee is ready!
  boiling...
  Coffee is ready!
  boiling...
  Coffee is ready!
  boiling...
  Coffee is ready!
  Please add water!
  Please add water!
  Blub blub blub...
  $>
\end{42console}

\begin{42console}
  $> cat machine.log
  (cmaxime)Running: Start Machine      [ exec-time = 0.001 ms ]
  (cmaxime)Running: Boil Water         [ exec-time = 0.005 ms ]
  (cmaxime)Running: Make Coffee        [ exec-time = 2.499 s  ]
  (cmaxime)Running: Start Machine      [ exec-time = 0.002 ms ]
  (cmaxime)Running: Boil Water         [ exec-time = 0.005 ms ]
  (cmaxime)Running: Make Coffee        [ exec-time = 2.618 s  ]
  (cmaxime)Running: Start Machine      [ exec-time = 0.003 ms ]
  (cmaxime)Running: Boil Water         [ exec-time = 0.004 ms ]
  (cmaxime)Running: Make Coffee        [ exec-time = 2.676 s  ]
  (cmaxime)Running: Start Machine      [ exec-time = 0.003 ms ]
  (cmaxime)Running: Boil Water         [ exec-time = 0.004 ms ]
  (cmaxime)Running: Make Coffee        [ exec-time = 2.648 s  ]
  (cmaxime)Running: Start Machine      [ exec-time = 0.011 ms ]
  (cmaxime)Running: Make Coffee        [ exec-time = 0.029 ms ]
  (cmaxime)Running: Start Machine      [ exec-time = 0.009 ms ]
  (cmaxime)Running: Make Coffee        [ exec-time = 0.024 ms ]
  (cmaxime)Running: Add Water          [ exec-time = 5.026 s  ]
  $>
\end{42console}

Pay attention, the length between ":" and "[" is 20].
Draw the corresponding conclusions on this part of a log entry.

% ===========================(fin ex 02)         %
% ============================================== %

\newpage

% ============================================== %
% ===========================(start ex 03)       %
\chapter{Exercise 03}
\extitle{Json issues}
\turnindir{ex03}
\exnumber{03}
\exfiles{csvreader.py}
\exforbidden{None}
\makeheaderfilesforbidden

% ================================= %
\section*{Objective}
% --------------------------------- %
The goal of this exercise is to implement a context manager as a class.
Thus you are strongly encouraged to do some research about context manager.

% ================================= %
\section*{Instructions}
% --------------------------------- %
Implement a \texttt{CsvReader} class that opens, reads, and parses a CSV file.
This class is then a context manager as class.
In order to create it, your class requires a few built-in methods:
\begin{itemize}
  \item \texttt{\_\_init\_\_},
  \item \texttt{\_\_enter\_\_},
  \item \texttt{\_\_exit\_\_}.
\end{itemize}
It is mandatory to close the file once the process has completed.
You are expected to handle properly badly formatted CSV file (i.e. handle the exception):
\begin{itemize}
  \item mistmatch between number of fields and number of records,
  \item records with different length.
\end{itemize}

\begin{minted}[bgcolor=darcula-back,formatcom=\color{lightgrey},fontsize=\scriptsize]{python}
  class CsvReader():
	def __init__(self, filename=None, sep=',', header=False, skip_top=0, skip_bottom=0):
		# ... Your code here ...

	def __enter__(...):
		# ... Your code here ...
	
	def __exit__(...):
		# ... Your code here ...
	
	def getdata(self):
	""" Retrieves the data/records from skip_top to skip bottom.
	Return:
		nested list (list(list, list, ...)) representing the data.
	"""
		# ... Your code here ...

	def getheader(self):
	""" Retrieves the header from csv file.
	Returns:
		list: representing the data (when self.header is True).
        None: (when self.header is False).
	"""
		# ... Your code here ...
\end{minted}

\texttt{CSV} (for Comma-Separated Values) file is a delimited text file which uses a comma to separate values.
Therefore, the field separator (or delimiter) is usually a comma (\texttt{,})
but with your context manager you have to offer the possibility to change this parameter.


One can decide if the class instance skips lines at the top and the bottom of the file via the
parameters \texttt{skip\_top} and \texttt{skip\_bottom}.
One should also be able to keep the first line as a header if \texttt{header} is \texttt{True}.


The file should not be corrupted (either a line with too many values or a line
with too few values), otherwise return \texttt{None}.\\
You have to handle the case \texttt{file not found}.\\


You are expected to implement two methods:
\begin{itemize}
  \item \texttt{getdata()},
  \item \texttt{getheader()}.
\end{itemize}

\begin{minted}[bgcolor=darcula-back,formatcom=\color{lightgrey},fontsize=\scriptsize]{python}
from csvreader import CsvReader

if __name__ == "__main__":
	with CsvReader('good.csv') as file:
		data = file.getdata()
		header = file.getheader()
\end{minted}

\begin{minted}[bgcolor=darcula-back,formatcom=\color{lightgrey},fontsize=\scriptsize]{python}
from csvreader import CsvReader

if __name__ == "__main__":
	with CsvReader('bad.csv') as file:
		if file == None:
			print("File is corrupted")
\end{minted}


% ===========================(fin ex 03)         %
% ============================================== %

\newpage

% ============================================== %
% ===========================(start ex 04)       %
\chapter{Exercise 04}
\extitle{MiniPack}
\turnindir{ex04}
\exnumber{04}
\exfiles{build.sh, *.py, *.md, *.cfg, *.txt}
\exforbidden{None}
\makeheaderfilesforbidden


% ================================= %
\section*{Objective}
% --------------------------------- %
The goal of the exercise is to learn how to build a package and understand the magnificence of \href{https://pypi.org/}{PyPi}.


% ================================= %
\section*{Instructions}
% --------------------------------- %
You have to create a package called \texttt{my\_minipack}.
\hint{\href{https://docs.python.org/3.9/distributing/index.html}{RTFM}}

It will have 2 \textbf{modules}: 
\begin{itemize}
  \item the progress bar (module00 ex10) which should be imported it via \texttt{import my\_minipack.progressbar},
  \item the logger (module02 ex02), which should be imported via \texttt{import my\_minipack.logger}.
\end{itemize}


The package will be installed via pip using one of the following commands (both should work):  
\begin{42console}
  $> pip install ./dist/my_minipack-1.0.0.tar.gz
  $> pip install ./dist/my_minipack-1.0.0-py3-none-any.whl
\end{42console}

Based on the following terminal commands and corresponding outputs, draw the necessary conclusion.

\begin{42console}
  $> python -m venv tmp_env && source tmp_env/bin/activate
  (tmp_env) > pip list
  # Ouput
  Package    Version
  ---------- -------
  pip        19.0.3 
  setuptools 40.8.0 

  (tmp_env) $> cd ex04/ && bash build.sh
  # Output ... No specific verbose expected, do as you wish ...
  ...
  (tmp_env) $> ls dist
  # Output
  my_minipack-1.0.0-py3-none-any.whl  my_minipack-1.0.0.tar.gz

  (tmp_env) $> pip list
  # Output
  Package     Version
  ----------- -------
  my-minipack 1.0.0
  pip         21.0.1 # the last version at the time
  setuptools  54.2.0 # the last version at the time
  wheel       0.36.2 # the last version at the time

  (tmp_env) $> pip show -v my_minipack
  # Ouput (minimum metadata asked)
  Name: my-minipack
  Version: 1.0.0
  Summary: Howto create a package in python.
  Home-page: None
  Author: mdavid
  Author-email: mdavid@student.42.fr
  License: GPLv3
  Location: [PATH TO BOOTCAMP PYTHON]/module02/tmp_env/lib/python3.7/site-packages
  Requires: 
  Required-by: 
  Metadata-Version: 2.1
  Installer: pip
  Classifiers:
  Development Status :: 3 - Alpha
  Intended Audience :: Developers
  Intended Audience :: Students
  Topic :: Education
  Topic :: HowTo
  Topic :: Package
  License :: OSI Approved :: GNU General Public License v3 (GPLv3)
  Programming Language :: Python :: 3
  Programming Language :: Python :: 3 :: Only
(tmp_env) $>
\end{42console}

Also add a LICENSE.md (you can choose a real license or a fake one it does not matter) and a README file where you write a small documentation about your library

The `build.sh` script upgrades `pip`, and \textbf{builds} the distribution packages in `wheel` and `egg` format.

\info{
You can ensure whether the package was properly installed by running the command \texttt{pip list}
that displays the list of installed packages and check the metadata of the package with
\texttt{pip show -v my\_minipack}.
Of course do not reproduce the exact same metadata, change the author information, modify the summary Topic and Audience items if you want to.
}
% ===========================(fin ex 04)         %
% ============================================== %
\newpage

% ============================================== %
% ===========================(start ex 05)       %
\chapter{Exercise 05}
\extitle{TinyStatistician}
\turnindir{ex05}
\exnumber{05}
\exfiles{TinyStatistician.py}
\exforbidden{Any function that calculates mean, median, quartiles, variance or standar deviation for you.}
\makeheaderfilesforbidden


% ================================= %
\section*{Objective}
% --------------------------------- %
Initiation to very basic statistic notions.

% ================================= %
\section*{Instructions}
% --------------------------------- %
Create a class named \texttt{TinyStatistician} that implements the following methods:
\begin{itemize}
  \item \texttt{mean(x)}: computes the mean of a given non-empty list or array \texttt{x}, using a for-loop.
  The method returns the mean as a float, otherwise \texttt{None} if \texttt{x} is an empty list or array.
  Given a vector \texttt{x} of dimension $m \times 1$, the mathematical formula of its mean is:
  $$
  \mu = \frac{\sum_{i = 1}^{m}{x_i}}{m}
  $$
  \item \texttt{median(x)}: computes the median of a given non-empty list or array \texttt{x}.
  The method returns the median as a float, otherwise \texttt{None} if \texttt{x} is an empty list or array. 
  \item \texttt{quartiles(x)}: computes the $1^{\text{st}}$ and $3^{\text{rd}}$ quartiles of a given non-empty array \texttt{x}.
  The method returns the quartile as a float, otherwise \texttt{None} if \texttt{x} is an empty list or array.
  \item \texttt{var(x)}: computes the variance of a given non-empty list or array \texttt{x}, using a for-loop.
  The method returns the variance as a float, otherwise \texttt{None} if \texttt{x} is an empty list or array.
  Given a vector \texttt{x} of dimension $m \times 1$, the mathematical formula of its variance is:
  $$
  \sigma^2 = \frac{\sum_{i = 1}^{m}{(x_i - \mu)^2}}{m} = \frac{\sum_{i = 1}^{m}{[x_i - (\frac{1}{m}\sum_{j = 1}^{m}{x_j}})]^2}{m}
  $$
  \item \texttt{std(x)} : computes the standard deviation of a given non-empty list  or array \texttt{x}, using a for-loop.
  The method returns the standard deviation as a float, otherwise \texttt{None} if \texttt{x} is an empty list or array.
  Given a vector \texttt{x} of dimension $m \times 1$, the mathematical formula of its standard deviation is:
  $$
  \sigma = \sqrt{\frac{\sum_{i = 1}^{m}{(x_i - \mu)^2}}{m}} = \sqrt{\frac{\sum_{i = 1}^{m}{[x_i - (\frac{1}{m}\sum_{j = 1}^{m}{x_j}})]^2}{m}}
  $$
\end{itemize}

All methods take a \texttt{list} or a \texttt{numpy.ndarray} as parameter.\\
We are assuming that all inputs have a correct format, i.e. a list or array of numeric type or empty list or array.
You don't have to protect your functions against input errors.


% ================================= %
\section*{Examples}
% --------------------------------- %
\begin{minted}[bgcolor=darcula-back,formatcom=\color{lightgrey},fontsize=\scriptsize]{python}
from TinyStatistician import TinyStatistician
tstat = TinyStatistician()
a = [1, 42, 300, 10, 59]

tstat.mean(a)
# Expected result: 82.4

tstat.median(a)
# Expected result: 42.0

tstat.quartile(a)
# Expected result: [10.0, 59.0]

tstat.var(a)
# Expected result: 12279.439999999999

tstat.std(a)
# Expected result: 110.81263465868862  
\end{minted}



% ===========================(fin ex 05)         %
% ============================================== %

\newpage

% ================================= %
\section*{Contact}
% --------------------------------- %
You can contact 42AI association by email: contact@42ai.fr\\
You can join the association on \href{https://join.slack.com/t/42-ai/shared_invite/zt-ebccw5r7-YPkDM6xOiYRPjqJXkrKgcA}{42AI slack}
and/or posutale to \href{https://forms.gle/VAFuREWaLmaqZw2D8}{one of the association teams}.

% ================================= %
\section*{Acknowledgements}
% --------------------------------- %
The modules Python \& ML is the result of a collective work, we would like to thanks:
\begin{itemize}
  \item Maxime Choulika (cmaxime),
  \item Pierre Peigné (ppeigne),
  \item Matthieu David (mdavid).
\end{itemize}
who supervised the creation, the enhancement and this present transcription.

\begin{itemize}
    \item Amric Trudel (amric@42ai.fr)
    \item Baptiste Lefeuvre (blefeuvr@student.42.fr)
    \item Mathilde Boivin (mboivin@student.42.fr)
    \item Tristan Duquesne (tduquesn@student.42.fr)
    \item Quentin Feuillade Montixi (qfeuilla@student.42.fr)
\end{itemize}
for your investment for the creation and development of these modules.

\begin{itemize}
    \item Barthélémy Leveque (bleveque@student.42.fr)
    \item Remy Oster (roster@student.42.fr)
    \item Quentin Bragard (qbragard@student.42.fr)
    \item Marie Dufourq (madufour@student.42.fr)
    \item Adrien Vardon (advardon@student.42.fr)
\end{itemize}
who betatest the first version of the modules of Machine Learning.
\vfill
\doclicenseThis

\end{document}