\chapter{Exercise 03}
\extitle{Json issues}
\turnindir{ex03}
\exnumber{03}
\exfiles{csvreader.py}
\exforbidden{None}
\makeheaderfilesforbidden

% ================================= %
\section*{Objective}
% --------------------------------- %
The goal of this exercise is to implement a context manager as a class.\\
Thus you are strongly encouraged to do some preliminary research about context manager.\\

% ================================= %
\section*{Instructions}
% --------------------------------- %
Implement a \texttt{CsvReader} class that opens, reads, and parses a CSV file.\\
\
This class is then a context manager as a class.\\
\
In order to create it, your class requires a few built-in methods:
\begin{itemize}
  \item \texttt{\_\_init\_\_},
  \item \texttt{\_\_enter\_\_},
  \item \texttt{\_\_exit\_\_}.
\end{itemize}
It is mandatory to close the file once the process is completed.
You are expected to handle properly badly-formatted CSV file (i.e. handle the exception):

\begin{itemize}
  \item mistmatch between number of fields and number of records,
  \item records with different lengths.
\end{itemize}

\begin{minted}[bgcolor=darcula-back,formatcom=\color{lightgrey},fontsize=\scriptsize]{python}
  class CsvReader():
	def __init__(self, filename=None, sep=',', header=False, skip_top=0, skip_bottom=0):
		# ... Your code here ...

	def __enter__(...):
		# ... Your code here ...
	
	def __exit__(...):
		# ... Your code here ...
	
	def getdata(self):
	""" Retrieves the data/records from skip_top to skip bottom.
	  Returns:
		  nested list (list(list, list, ...)) representing the data.
	"""
		# ... Your code here ...

	def getheader(self):
	""" Retrieves the header from the csv file.
	Returns:
		list: representing the data (when self.header is True).
    None: (when self.header is False).
	"""
		# ... Your code here ...
\end{minted}
\newline
\texttt{CSV} (for Comma-Separated Values) files are delimited text files which use a given character to separate values.\\
\\
The separator (or delimiter) is usually a comma (\texttt{,}) or an hyphen comma (\texttt{;}), 
but with your context manager you have to offer the possibility to change this parameter.\\
\\
One can decide if the class instance skips lines at the top and the bottom of the file via the
parameters \texttt{skip\_top} and \texttt{skip\_bottom}.\\
\\
One should also be able to keep the first line as a header if \texttt{header} is \texttt{True}.\\
\\
The file should not be corrupted (either a line with too many values or a line
with too few values), otherwise return \texttt{None}.\\
\\
You have to handle the case \texttt{file not found}.\\
\\
You are expected to implement two methods:
\begin{itemize}
  \item \texttt{getdata()},
  \item \texttt{getheader()}.
\end{itemize}

\begin{minted}[bgcolor=darcula-back,formatcom=\color{lightgrey},fontsize=\scriptsize]{python}
from csvreader import CsvReader

if __name__ == "__main__":
	with CsvReader('good.csv') as file:
		data = file.getdata()
		header = file.getheader()
\end{minted}

\begin{minted}[bgcolor=darcula-back,formatcom=\color{lightgrey},fontsize=\scriptsize]{python}
from csvreader import CsvReader

if __name__ == "__main__":
	with CsvReader('bad.csv') as file:
		if file == None:
			print("File is corrupted")
\end{minted}