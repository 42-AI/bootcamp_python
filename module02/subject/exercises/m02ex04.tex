\chapter{Exercise 04}
\extitle{MiniPack}
\turnindir{ex04}
\exnumber{04}
\exfiles{build.sh, *.py, *.md, *.cfg, *.txt}
\exforbidden{None}
\makeheaderfilesforbidden


% ================================= %
\section*{Objective}
% --------------------------------- %
The goal of this exercise is to learn how to build a package and understand the magnificence of \href{https://pypi.org/}{PyPi}.

% ================================= %
\section*{Instructions}
% --------------------------------- %
You have to create a package called \texttt{my\_minipack}.\\
\hint{\href{https://docs.python.org/3.9/distributing/index.html}{RTFM}}
It will have 2 \textbf{modules}: 
\begin{itemize}
  \item the progress bar (module00 ex10) which should be imported it via \texttt{import my\_minipack.progressbar},
  \item the logger (module02 ex02), which should be imported via \
    \texttt{import my\_minipack.logger}.
\end{itemize}
The package will be installed via pip using one of the following commands (both should work):  
\begin{42console}
  $> pip install ./dist/my_minipack-1.0.0.tar.gz
  $> pip install ./dist/my_minipack-1.0.0-py3-none-any.whl
\end{42console}
\
Based on the following terminal commands and corresponding outputs, draw the necessary conclusion.
\
\begin{42console}
  $> python -m venv tmp_env && source tmp_env/bin/activate
  (tmp_env) > pip list
  # Ouput
  Package    Version
  ---------- -------
  pip        19.0.3 
  setuptools 40.8.0 

  (tmp_env) $> cd ex04/ && bash build.sh
  # Output ... No specific verbose expected, do as you wish ...
  ...
  (tmp_env) $> ls dist
  # Output
  my_minipack-1.0.0-py3-none-any.whl  my_minipack-1.0.0.tar.gz

  (tmp_env) $> pip list
  # Output
  Package     Version
  ----------- -------
  my-minipack 1.0.0
  pip         21.0.1 # the last version at the time
  setuptools  54.2.0 # the last version at the time
  wheel       0.36.2 # the last version at the time

  (tmp_env) $> pip show -v my_minipack
  # Ouput (minimum metadata asked)
  Name: my-minipack
  Version: 1.0.0
  Summary: Howto create a package in python.
  Home-page: None
  Author: mdavid
  Author-email: mdavid@student.42.fr
  License: GPLv3
  Location: [PATH TO BOOTCAMP PYTHON]/module02/tmp_env/lib/python3.7/site-packages
  Requires: 
  Required-by: 
  Metadata-Version: 2.1
  Installer: pip
  Classifiers:
  Development Status :: 3 - Alpha
  Intended Audience :: Developers
  Intended Audience :: Students
  Topic :: Education
  Topic :: HowTo
  Topic :: Package
  License :: OSI Approved :: GNU General Public License v3 (GPLv3)
  Programming Language :: Python :: 3
  Programming Language :: Python :: 3 :: Only
(tmp_env) $>
\end{42console}
Also add a LICENSE.md (you can choose a real license or a fake one it does not matter) and a README file where you will write a small documentation about your packaged library.
\\
The `build.sh` script upgrades `pip`, and \textbf{builds} the distribution packages in `wheel` and `egg` formats.
\\
\info{
You can check whether the package was properly installed by running the command \texttt{pip list}
that displays the list of installed packages and check the metadata of the package with
\texttt{pip show -v my\_minipack}.
Of course do not reproduce the exact same metadata, change the author information, modify the summary Topic and Audience items if you want to.
}