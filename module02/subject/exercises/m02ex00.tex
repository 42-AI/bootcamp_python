\chapter{Exercise 00}
\extitle{Map, filter, reduce}
\turnindir{ex00}
\exnumber{00}
\exfiles{ft\_map.py, ft\_filter.py, ft\_reduce.py}
\exforbidden{map, filter, reduce}
\makeheaderfilesforbidden

% ================================== %
\section*{Objective}
% ---------------------------------- %
The goal of this exercise is to work on the built-in functions \texttt{map},
\texttt{filter} and \texttt{reduce}.

% ================================== %
\section*{Instructions}
% ---------------------------------- %
Implement the functions \texttt{ft\_map}, \texttt{ft\_filter} and \texttt{ft\_reduce}.\\
\\
Take the time to understand the use cases of these two built-in functions
(\texttt{map} and \texttt{filter}) and the function \texttt{reduce} in the functools module.\\
\\
You are not expected to code specific classes to create \texttt{ft\_map},
\texttt{ft\_filter} or \texttt{ft\_reduce} objects, take a closer look
at the examples to know what to do.\\
\\
Here are the signatures of the functions:\
\\
\begin{minted}[bgcolor=darcula-back,formatcom=\color{lightgrey},fontsize=\scriptsize]{python}
def ft_map(function_to_apply, iterable):
	"""Map the function to all elements of the iterable.
	Args:
	  function_to_apply: a function taking an iterable.
	  iterable: an iterable object (list, tuple, iterator).
	Return:
	  An iterable.
	  None if the iterable can not be used by the function.
	"""
  # ... Your code here ...

def ft_filter(function_to_apply, iterable):
	"""Filter the result of function apply to all elements of the iterable.
	Args:
	  function_to_apply: a function taking an iterable.
	  iterable: an iterable object (list, tuple, iterator).
	Return:
	  An iterable.
	  None if the iterable can not be used by the function.
	"""
  # ... Your code here ...

def ft_reduce(function_to_apply, iterable):
	"""Apply function of two arguments cumulatively.
	Args:
	  function_to_apply: a function taking an iterable.
	  iterable: an iterable object (list, tuple, iterator).
	Return:
	  A value, of same type of elements in the iterable parameter.
	  None if the iterable can not be used by the function.
	"""
  # ... Your code here ...
\end{minted}

% ================================== %
\section*{Examples}
% ---------------------------------- %

\begin{minted}[bgcolor=darcula-back,formatcom=\color{lightgrey},fontsize=\scriptsize]{python}
# Example 1:
x = [1, 2, 3, 4, 5]
ft_map(lambda dum: dum + 1, x)
# Output:
<generator object ft_map at 0x7f708faab7b0> # The adress will be different

list(ft_map(lambda t: t + 1, x))
# Output:
[2, 3, 4, 5, 6]

# Example 2:
ft_filter(lambda dum: not (dum % 2), x)
# Output:
<generator object ft_filter at 0x7f709c777d00> # The adress will be different

list(ft_filter(lambda dum: not (dum % 2), x))
# Output:
[2, 4]

# Example 3:
lst = ['H', 'e', 'l', 'l', 'o', ' ', 'w', 'o', 'r', 'l', 'd']
ft_reduce(lambda u, v: u + v, lst)
# Output:
"Hello world"
\end{minted}
\\
You are expected to raise similar exceptions than those of
\texttt{map}, \texttt{filter} and \texttt{reduce} when wrong parameters are given (but no need
to reproduce the exact same exception messages).